\newpage
\drawBar{Zetten}

\hoofdstuk{5}{Zetten}
\label{sec:zettenLang}

\customBoxItalic{Een zet in Fritsen bestaat uit het \ul{open} opleggen van \'e\'en of twee kaarten op \'e\'en van de \ul{open} stapels. De \huidigeSpeler kan kiezen tussen de zetten beschreven in dit hoofdstuk. Afhankelijk van het type zet moet de \huidigeSpeler of moeten de \andereSpelers mogelijk drinken.}

\beginLijstKlein{5}
\item Een kaart op een kaart van hetzelfde symbool met een lagere waarde leggen. Bijvoorbeeld, een \kaart{harten boer}, \kaart{harten vrouw}, \kaart{harten koning} en \kaart{harten aas} mogen op een \kaart{harten 10}.
\label{zet:hoger}
\eindLijst{}

\vervolgLijstKlein{}
\item \textbf{Offerfrits}: Een kaart op een kaart van hetzelfde symbool met een waarde van \'e\'en hoger leggen. Bijvoorbeeld, een \kaart{harten 3} mag op een \kaart{harten 4}. \\De \huidigeSpeler moet:
\puntLijst{}
\item Proosten op \quotes{offer Frits} en \textbf{Fritsen}.
\eindPuntLijst{}
\label{zet:offer_frits}
\eindLijst{}

\vervolgLijstKlein{}
\item Een \kaart{2} op een \kaart{aas} van hetzelfde symbool leggen. Bijvoorbeeld, de \kaart{harten 2} mag op een \kaart{harten aas}.
\eindLijst{}

\customBox{\textit{Zet} \ref{zet:dubbel} \textit{is de enige zet waar twee kaarten tegelijkertijd opgelegd mogen worden.}}

\vervolgLijstKlein{}
\item \label{zet:dubbel} Een \kaart{koning} en een \kaart{aas} van hetzelfde symbool op een vrouw van hetzelfde symbool leggen. Bijvoorbeeld, een \kaart{harten koning} en een \kaart{harten aas} mogen achtereenvolgend op een \kaart{harten vrouw}.
\eindLijst{}

\vervolgLijstKlein{}
\item \label{zet:lisa} \textbf{Lisaatje}: Een \kaart{Frits} (\'e\'en van de \kaart{koningen}) op een \kaart{Lisa} (de \kaart{harten aas}) leggen en andersom. De \andereSpelers moeten:
\puntLijst{}
\item Proosten op \quotes{kut Lisa}
\item \textbf{Een Lisaatje nemen}\footnotemark[1].
\eindPuntLijst{}
\eindLijst{}

\vervolgLijstKlein{}
\item \label{zet:kim} \textbf{Kimmetje}: Een \kaart{Kim} (\'e\'en van de \kaart{vrouwen}) op een \kaart{Kim} leggen. \\De \andereSpelers moeten:
\puntLijst{}
\item Proosten op \quotes{kut Kim}
\item Één van de volgende handelingen uitvoeren:
\numeriekeLijst{}
\item \textbf{Een Kimmetje nemen}\footnotemark[1] als de opgelegde kaart \ul{niet} op een stapel ligt met \\achtereenvolgend drie \kaart{Kimmen}.
\item \textbf{Een dubbele Kim nemen}\footnotemark[2] als de opgelegde kaart \ul{wel} op een stapel ligt met \\achtereenvolgend drie \kaart{Kimmen}.
\eindNumeriekeLijst{}
\eindPuntLijst{}
\eindLijst{}

\footnotetext[1]{\textbf{een Lisaatje nemen} en \textbf{een Kimmetje nemen} zijn equivalent aan \textbf{Fritsen} (zie definitie \ref{item:enkel_fritsen_equivalent})}
\footnotetext[2]{\textbf{een dubbele Kim nemen} is equivalent aan \textbf{Dubbelfritsen} (zie definitie \ref{item:dubbelfritsen_equivalent} op pagina \pageref{item:dubbelfritsen_equivalent})}

\newpage
\drawBar{Zetten - Vervolg}
\deelhoofdstuk{Zetten - Vervolg}
\label{sec:zettenLang_2}

\vervolgLijstKlein{}
\item \label{zet:joris} \textbf{Jorisje}: Een \kaart{Chantal} (\'e\'en van de \kaart{rode vrouwen}) op een \kaart{Joris} (de \kaart{klaveren boer}) leggen of andersom. \\De \andereSpelers moeten:
\puntLijst{}
\item Proosten op \quotes{Chantal} en \textbf{een Jorisje nemen}\footnotemark[1].
\eindPuntLijst{}
\eindLijst{}

\customBox{\textit{Het neerleggen van je laatste kaart is met zetten \ref{zet:thierry}, \ref{zet:caroline} en \ref{zet:joker_1} \textit{en} \ref{zet:joker_2} niet toegestaan.}}

\vervolgLijstKlein{}
\item \textbf{Thierry'tje}: Een \kaart{Thierry} (\'e\'en van de \kaart{zessen}) op een \kaart{vrouw} leggen. \\Voer vervolgens de volgende handelingen in de gegeven volgorde uit:
\puntLijst{}
\item De \huidigeSpeler proost op \quotes{Baudet} en \textbf{neemt een Thierry'tje}\footnotemark[2].
\item \Frits neemt alle kaarten van de \huidigeSpeler \ul{dicht} in ontvangst.
\item \Frits telt het aantal gekregen kaarten.
\item \Frits legt de gekregen kaarten \ul{dicht} onderop de \ul{dichte} stapel.
\item \Frits geeft de \huidigeSpeler hetzelfde aantal nieuwe \ul{dichte} kaarten terug van bovenop de \ul{dichte} stapel.
\eindPuntLijst{}
\label{zet:thierry}
\eindLijst{}

\vervolgLijstKlein{}
\item \textbf{Caroline'tje}: Een \kaart{6} op een \kaart{Caroline} (\'e\'en van de \kaart{boeren}) leggen. \\De \huidigeSpeler moet:
\puntLijst{}
\item Proosten op \quotes{van der Plas} en \textbf{een Caroline'tje nemen}\footnotemark[3].
\item Één van de volgende handelingen uitvoeren:
\numeriekeLijst{}
\item Een shotglaasje op \'e\'en van de \ul{open} \textbf{stapels} plaatsen als er zich nog geen shotglaasje op \'e\'en van de \ul{open} stapels bevindt.
\item Het \ul{huidige} shotglaasje dat zich op \'e\'en van de \ul{open} \textbf{stapels} bevindt verplaatsen naar een andere \ul{open} stapel.
\item Het \ul{huidige} shotglaasje dat zich op \'e\'en van de \ul{open} \textbf{stapels} bevindt \ul{niet} verplaatsen.
\eindNumeriekeLijst{}
\eindPuntLijst{}
\label{zet:caroline}
\eindLijst{}

\vervolgLijstKlein{}
\item \label{zet:joker_1} Een \kaart{joker} op de \ul{bestaande} jokerstapel leggen. \\De \andereSpelers moeten:
\puntLijst{}
\item Proosten op \quotes{Frits} en \textbf{Fritsen}.
\eindPuntLijst{}
\eindLijst{}

\vervolgLijstKlein{}
\item \label{zet:joker_2} Een jokerstapel beginnen als er nog \ul{geen} jokerstapel ligt. \\De \andereSpelers moeten:
\puntLijst{}
\item Proosten op \quotes{Frits} en \textbf{Fritsen}.
\eindPuntLijst{}
\eindLijst{}

\vervolgLijstKlein{}
\item Een \kaart{9} op een kaart ongelijk aan een \kaart{9} of een \kaart{joker} leggen.
\eindLijst{}

\vervolgLijstKlein{}
\item Een \kaart{9} op een \kaart{9} leggen. \\De \andereSpelers moeten:
\puntLijst{}
\item Proosten op \quotes{iedereen dubbelfrits} en \textbf{een dubbele Frits nemen}.
\eindPuntLijst{}
\eindLijst{}



\footnotetext[1]{\textbf{een Jorisje nemen} is equivalent aan \textbf{Fritsen} (zie definitie \ref{item:enkel_fritsen_equivalent} op pagina  \pageref{item:enkel_fritsen_equivalent})}
\footnotetext[2]{\textbf{een Thierry'tje nemen} is equivalent aan \textbf{Dubbelfritsen} (zie definitie \ref{item:dubbelfritsen_equivalent} op pagina \pageref{item:dubbelfritsen_equivalent})}
\footnotetext[3]{\textbf{een Caroline'tje nemen} is equivalent aan \textbf{Dubbelfritsen} (zie definitie \ref{item:dubbelfritsen_equivalent} op pagina \pageref{item:dubbelfritsen_equivalent})}

