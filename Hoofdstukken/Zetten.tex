\newpage
\drawBar{Zetten}

\hoofdstuk{6}{Zetten}
\label{sec:zettenLang}

\customBoxItalic{Een zet in Fritsen bestaat uit het \ul{open} opleggen van kaart(en) op \'e\'en van de \ul{open} stapels. De \huidigeSpeler kan kiezen tussen de zetten beschreven in dit hoofdstuk. Afhankelijk van het type zet moet de \huidigeSpeler of moeten de \andereSpelers mogelijk drinken.}

\beginLijst{6}
    \item Een kaart op een kaart van hetzelfde symbool met een lagere waarde leggen. Bijvoorbeeld, een \kaart{harten boer}, \kaart{harten vrouw}, \kaart{harten koning} en \kaart{harten aas} mogen op een \kaart{harten 10}.
\eindLijst{}

\vervolgLijst{}
    \item \textbf{Offerfrits}: Een kaart op een kaart van hetzelfde symbool met een waarde van \'e\'en hoger leggen. Bijvoorbeeld, een \kaart{harten 3} mag op een \kaart{harten 4}. \\De \huidigeSpeler moet:
    \puntLijst{}
        \item Proosten op \quotes{offerfrits} en \textbf{Fritsen}.
    \eindPuntLijst{}
    \label{zet:offer_frits}
\eindLijst{}

\vervolgLijst{}
    \item Een \kaart{2} op een \kaart{aas} van hetzelfde symbool leggen. Bijvoorbeeld, de \kaart{harten 2} mag op een \kaart{harten aas}.
\eindLijst{}

\customBox{\textit{Zet} \ref{zet:dubbel} \textit{is de enige zet in Fritsen waar \textbf{twee} kaarten tegelijkertijd opgelegd mogen worden.}}

\vervolgLijst{}
    \item \label{zet:dubbel} Een \kaart{koning} en een \kaart{aas} van hetzelfde symbool achtereenvolgend op een vrouw van hetzelfde symbool leggen. Bijvoorbeeld, een \kaart{harten koning} en een \kaart{harten aas} mogen achtereenvolgend op een \kaart{harten vrouw}.
\eindLijst{}

\vervolgLijst{}
    \item \label{zet:lisa} \textbf{Lisaatje}: Een \kaart{Frits} (\'e\'en van de \kaart{koningen}) op een \kaart{Lisa} (de \kaart{harten aas}) leggen en andersom. De \andereSpelers moeten:
     \puntLijst{}
        \item Één van de volgende acties uitvoeren:
        \numeriekeLijst{}
            \item Proosten op \quotes{kut Lisa} als \textbf{[Lisa]}\footnotemark[1] er \ul{niet} bij is.
            \item Proosten op \quotes{super mooi Lisa} als \textbf{[Lisa]}\footnotemark[1] er \ul{wel} bij is.
        \eindNumeriekeLijst{}
        \item \textbf{Een Lisaatje nemen}\footnotemark[2].
    \eindPuntLijst{}
\eindLijst{}

\vervolgLijst{}
    \item \label{zet:kim} \textbf{Kimmetje}: Een \kaart{Kim} (\'e\'en van de \kaart{vrouwen}) op een \kaart{Kim} leggen. \\De \andereSpelers moeten:
    \puntLijst{}
    \item Één van de volgende acties uitvoeren:
        \numeriekeLijst{}
            \item Proosten op \quotes{kut Kim} als \textbf{[Kim]}\footnotemark[3] er \ul{niet} bij is.
            \item Proosten op \quotes{super mooi Kim} als \textbf{[Kim]}\footnotemark[3] er \ul{wel} bij is.
        \eindNumeriekeLijst{}
    \item Één van de volgende acties uitvoeren:
        \numeriekeLijst{}
            \item \textbf{Een Kimmetje nemen}\footnotemark[2] als de opgelegde kaart \ul{niet} op een stapel ligt met \\achtereenvolgend drie \kaart{Kimmen}.
            \item \textbf{Een dubbele Kim nemen}\footnotemark[4] als de opgelegde kaart \ul{wel} op een stapel ligt met \\achtereenvolgend drie \kaart{Kimmen}.
        \eindNumeriekeLijst{}
    \eindPuntLijst{}
\eindLijst{}  

\footnotetext[1]{zie definitie \ref{item:lisa}} 
\footnotetext[2]{\textbf{een Lisaatje nemen} en \textbf{een Kimmetje nemen} zijn equivalent aan \textbf{Fritsen} (zie definitie \ref{item:enkel_fritsen_equivalent})}
\footnotetext[3]{zie definitie \ref{item:kim}} 
\footnotetext[4]{\textbf{een dubbele Kim nemen} is equivalent aan \textbf{dubbelfritsen} (zie definitie \ref{item:dubbelfritsen_equivalent})}

\newpage
\drawBar{Zetten - Vervolg}
\deelhoofdstuk{Zetten - Vervolg}
\label{sec:zettenLang_2}

\vervolgLijst{}
    \item \label{zet:joris} \textbf{Jorisje}: Een \kaart{Chantal} (\'e\'en van de \kaart{rode vrouwen}) op een \kaart{Joris} (de \kaart{klaveren boer}) leggen of andersom. \\De \andereSpelers moeten:
    \puntLijst{}
        \item Proosten op \quotes{Chantal} en \textbf{een Jorisje nemen}\footnotemark[1].
    \eindPuntLijst{}
\eindLijst{} 

\customBox{Het neerleggen van je laatste kaart is met zetten \ref{zet:thierry}\textit{,} \ref{zet:joker_1} \textit{en} \ref{zet:joker_2} \textit{niet toegestaan.}}

\vervolgLijst{}
    \item \label{zet:thierry} \textbf{Thierry'tje}: Een \kaart{Thierry} (\'e\'en van de \kaart{zessen}) op een \kaart{vrouw} leggen. \\Voer vervolgens de volgende acties in de gegeven volgorde uit:
    \puntLijst{}
        \item De \huidigeSpeler proost op \quotes{Baudet} en \textbf{neemt een Thierry'tje}\footnotemark[2].
        \item Vervolgens worden de volgende acties in de gegeven volgorde uitgevoerd als de \ul{dichte} stapel bestaat uit minimaal het aantal kaarten van de \textbf{[huidige speler]}:
        \puntLijst{}
            \item De \huidigeSpeler geeft zijn/haar kaarten aan \textbf{[Frits]}. \
            \item \Frits telt het aantal kaarten en geeft precies dit aantal nieuwe /ul{dichte} kaarten terug van bovenop de \ul{dichte} stapel. 
            \item \Frits legt de \ul{oude} kaarten onderop de \ul{dichte} stapel.
        \eindPuntLijst{}
    \eindPuntLijst{}
    \label{zet:thierry}
\eindLijst{} 

\vervolgLijst{}
    \item \label{zet:joker_1} Een \kaart{joker} op de \ul{bestaande} jokerstapel leggen. \\De \andereSpelers moeten:
    \puntLijst{}
        \item Proosten op \quotes{Frits} en \textbf{Fritsen}.
    \eindPuntLijst{}
\eindLijst{} 

\vervolgLijst{}
    \item \label{zet:joker_2} Een jokerstapel beginnen als er nog \ul{geen} jokerstapel ligt. \\De \andereSpelers moeten:
    \puntLijst{}
        \item Proosten op \quotes{Frits} en \textbf{Fritsen}.
    \eindPuntLijst{}
\eindLijst{} 

\vervolgLijst{}
    \item Een \kaart{9} op een kaart ongelijk aan een \kaart{9} of een \kaart{joker} leggen.
\eindLijst{} 

\vervolgLijst{}
    \item Een \kaart{9} op een \kaart{9} leggen. \\De \andereSpelers moeten:
    \puntLijst{}
        \item Proosten op \quotes{iedereen dubbel Frits} en \textbf{een dubbele Frits nemen}.
    \eindPuntLijst{}
\eindLijst{} 



\footnotetext[1]{\textbf{een Jorisje nemen} is equivalent aan \textbf{Fritsen} (zie definitie \ref{item:enkel_fritsen_equivalent})}
\footnotetext[2]{\textbf{een Thierry'tje nemen} is equivalent aan \textbf{Dubbelfritsen} (zie definitie \ref{item:dubbelfritsen_equivalent})}

