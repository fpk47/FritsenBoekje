\newpage
\drawBar{}

\hoofdstuk{2}{Regels}

\begin{tabular}{ll}
    \large{$\rightarrow$ Algemeen} & \hspace{0.5cm} \large{pagina \pageref{sec:algemeen_start} t/m \pageref{sec:algemeen_einde}} \\
    \large{$\rightarrow$ Beginfase} van het Spel & \hspace{0.5cm} \large{pagina \pageref{sec:beginfase_start} t/m \pageref{sec:beginfase_einde}}\\
    \large{$\rightarrow$ Vuil Fritsen} & \hspace{0.5cm} \large{pagina \pageref{sec:vuil_fritsen}}\\
    \large{$\rightarrow$ Beurten en Zetten} & \hspace{0.5cm} \large{pagina \pageref{sec:beurten_en_zetten_start} t/m \pageref{sec:beurten_en_zetten_einde}} \\
    \large{$\rightarrow$ Jokers} & \hspace{0.5cm} \large{pagina \pageref{sec:jokers}}\\
    \large{$\rightarrow$ Thierry Baudet en Jesse Klaver} & \hspace{0.5cm} \large{pagina \pageref{sec:thierry}}
\end{tabular}

\deelhoofdstuk{Regels - Algemeen}
\label{sec:algemeen_start}
\beginLijst{2}
    \item Als \textbf{[een speler]}, tenzij elders beschreven, een regel in dit document overtreedt, moet hij/zij:
    \puntLijst{}
        \item Proosten op \quotes{Frits} en \textbf{een Fritsje des nemen}\footnotemark[1].
    \eindPuntLijst{}
\eindLijst{}  

\vervolgLijst{}
    \item Als \eenSpeler het verboden woord \ul{\texttt{ADTEN}} zegt, moet hij/zij:
    \puntLijst{}
    \item Proosten op \quotes{Frits} en \textbf{een Fritsje des nemen}\footnotemark[1]. 
    \eindPuntLijst{}
\eindLijst{}   

\vervolgLijst{}
    \item Als een regel binnen de huidige situatie onduidelijk is of op meerdere manieren geïnterpreteerd kan worden, moet meer dan de helft van \textbf{[alle spelers]} overeenkomen tot een interpretatie van de desbetreffende regel.
\eindLijst{}

\vervolgLijst{}
    \item \label{item:beslissingCriteria} Als meer dan de helft van \textbf{[alle spelers]} het niet eens kan worden over een fritsgerelateerde kwestie gelden de volgende beslissingscriteria (in volgorde van belangrijkheid):
    \numeriekeLijst{}
        \item De beslissing wordt genomen door de grootste groep van \textbf{[alle spelers]} met dezelfde mening over de gerelateerde kwestie. 
        \item De beslissing wordt genomen door \Frits mits hij/zij bepaald is en hij/zij dit wilt.
        \item De beslissing wordt genomen door \Willem mits hij/zij bepaald is en hij/zij dit wilt.
        \item De beslissing wordt genomen door de oudste van \textbf{[alle spelers]}.
    \eindNumeriekeLijst{}
\eindLijst{}

\vervolgLijst{}
    \item Als \eenSpeler volgens meer dan de helft van \textbf{[alle spelers]} een domme of onnodige fritsgerelateerde fout maakt, moet hij/zij:
    \puntLijst{}
        \item Proosten op \quotes{Frits} en \textbf{een Fritsje des nemen}\footnotemark[1].
    \eindPuntLijst{}
\eindLijst{}  

\vervolgLijst{}
    \item Voor elke handeling, gebrek aan een handeling of verkeerde uitspraak mag \eenSpeler maximaal twee keer opgelegd worden om te drinken.
\eindLijst{}   

\vervolgLijst{}
    \item Als meer dan de helft van \textbf{[alle spelers]} goedkeurt dat \eenSpeler niet hoeft te drinken, hoeft hij/zij niet te drinken.
\eindLijst{}   

\vervolgLijst{}
    \item Het meer \textbf{Fritsen} dan nodig wordt niet beloond maar wel gewaardeerd.
\eindLijst{}   

\vervolgLijst{}
    \item Als \eenSpeler niet bezig is met een fritsgerelateerde taak, wordt het gewaardeerd als hij/zij lege shotglaasjes vult met de drank die op dat moment gebruikt wordt om te Fritsen.
\eindLijst{}   

\vervolgLijst{}
    \item Als \eenSpeler proost bij het uitvoeren van een fritsgerelateerde taak, moet hij/zij dit hoor- en verstaanbaar voor minimaal twee derde van \alleSpelers doen.
\eindLijst{} 

\vervolgLijst{}
    \item \EenSpeler hoeft niet de waarheid te spreken als hij/zij zegt dat hij/zij \textbf{uitfritsgarantie}\footnotemark[2] heeft.
\eindLijst{}   

\vervolgLijst{}
    \item \EenSpeler mag \medeSpelers niet bewust blokkeren. 
    \label{regel:speler_blokkeren}
\eindLijst{}   

\footnotetext[1]{\textbf{een Fritsje des nemen} is equivalent aan \textbf{Fritsen} (zie definitie \ref{item:enkel_fritsen_equivalent})}
\footnotetext[2]{zie definitie \ref{item:uitfritsgarantie}}

\newpage
\drawBar{Regels - Vervolg}
\deelhoofdstuk{Regels - Algemeen - Vervolg}

\customBoxItalic{Je kunt gewoon naar het toilet en bevorderende fritsgerelateerde handeling uitvoeren. In deze gevallen kan het spel worden gepauzeerd: niemand mag dan een zet doen.}

\vervolgLijst{}
    \item Het spel wordt gepauzeerd als \eenSpeler hoor- en verstaanbaar voor minimaal twee derde van \alleSpelers duidelijk maakt dat hij/zij naar het toilet wil en daarna binnen 9 seconden daadwerkelijk gaat.
    \label{regel:stilleggen_1}
\eindLijst{}

\vervolgLijst{}
    \item Het spel wordt gepauzeerd als \eenSpeler hoor- en verstaanbaar voor minimaal twee derde van \alleSpelers duidelijk maakt dat hij/zij drank wil halen die op dat moment gebruikt wordt om te Fritsen en daarna de desbetreffende drank binnen 9 seconden daadwerkelijk gaat halen.
    \label{regel:stilleggen_2}
\eindLijst{}

\vervolgLijst{}
    \item Het spel wordt gepauzeerd als \eenSpeler een handeling wil en gaat uitvoeren die het huidige potje Fritsen bevordert die door meer dan helft van \alleSpelers wordt goedgekeurd.
    \label{regel:stilleggen_3}
\eindLijst{}

\vervolgLijst{}
    \item Het spel wordt hervat wanneer \alleSpelers terug zijn na het uitvoeren van de handelingen beschreven in regels \ref{regel:stilleggen_1}, \ref{regel:stilleggen_2} en \ref{regel:stilleggen_3}. 
    \label{regel:stilleggen_4}
\eindLijst{} 

\customBoxItalic{Pak alleen kaarten van de dichte stapel en géén kaarten van de open stapels.}

\vervolgLijst{}
    \item Als \eenSpeler in de \ul{dichte} \textbf{stapel} kijkt, moet hij/zij in de gegeven volgorde:
    \puntLijst{}
        \item Proosten op \quotes{vierdubbel Frits} en \ul{twee keer} \textbf{een dubbele Frits nemen}\footnotemark[2]. 
        \item De \ul{dichte} \textbf{stapel} goed en zichtbaar schudden volgens meer dan de helft van \\ \textbf{[alle spelers]}.
        \item De \ul{dichte} \textbf{stapel} op de plek terugleggen waar deze lag voor het schudden
        \item Twee \ul{dichte} kaarten pakken van bovenop de \ul{dichte} \textbf{stapel}.
    \eindPuntLijst{}
    \label{regel:kijken_in_dichte_stapel}
\eindLijst{}  

\vervolgLijst{}
    \item Als \eenSpeler één kaart of meerdere kaarten van één van de \ul{open} \textbf{stapels} pakt en niet regel \ref{regel:kaarten_terugnemen_1}, \ref{regel:kaarten_terugnemen_2}, \ref{regel:kaarten_terugnemen_3} of \ref{regel:kaarten_terugnemen_4} aan het toepassen is, moet hij/zij:
    \puntLijst{}
       \item Proosten op \quotes{Frits} en \textbf{een Fritsje des nemen}\footnotemark[3].
        \item De net gepakte kaart(en) in dezelfde volgorde \ul{open} terugleggen op de desbetreffende \ul{open} \textbf{stapel}.
    \eindPuntLijst{}
\eindLijst{}  

\customBoxItalic{Laat niemand in je kaarten kijken en, wanneer gevraagd, vertel hoeveel je er hebt.}

\vervolgLijst{}
    \item \EenSpeler mag te allen tijde in de kaarten van \alleSpelers kijken.
\eindLijst{}

\vervolgLijst{}
    \item \EenSpeler mag niet de kaarten aanraken van zijn/haar \medeSpelers die bezig is met een fritsgerelateerde taak of die door hem/haar vastgehouden worden.
\eindLijst{}  

\vervolgLijst{}
    \item Als \eenSpeler kaarten vast heeft van zijn/haar \textbf{[medespelers]}, moet hij/zij deze binnen 9 seconden teruggeven.
\eindLijst{}

\vervolgLijst{}
    \item \EenSpeler is verplicht naar waarheid te vertellen hoeveel kaarten hij/zij in zijn/haar hand heeft wanneer dit gevraagd wordt door zijn/haar \textbf{[medespelers]}.
\eindLijst{}   

\footnotetext[1]{\textbf{een Fritsje des nemen} is equivalent aan \textbf{Fritsen} (zie definitie \ref{item:enkel_fritsen_equivalent})}
\footnotetext[2]{\textbf{een dubbele Frits nemen} is equivalent aan twee keer \textbf{Fritsen} (zie definities \ref{item:enkel_fritsen_equivalent} en \ref{item:dubbelfritsen_equivalent})}

\newpage
\drawBar{Regels - Vervolg}
\deelhoofdstuk{Regels - Algemeen - Vervolg}
\label{sec:algemeen_einde}


\customBoxItalic{Ga zorgvuldig met de shotglaasjes, regelboekjes en kaarten om.}

\vervolgLijst{}
    \item \EenSpeler mag geen shotglaasje omgooien.
\eindLijst{} 

\vervolgLijst{}
    \item \EenSpeler mag geen shotglaasje op een kaart zetten.
\eindLijst{} 

\vervolgLijst{}
    \item \EenSpeler mag geen fritsgerelateerde documenten natmaken.
\eindLijst{} 

\vervolgLijst{}
    \item \EenSpeler mag op fritsgerelateerde documenten alleen regelgerelateerde zaken schrijven die fritsgerelateerd zijn.
\eindLijst{}

\vervolgLijst{}
    \item \EenSpeler mag geen kaarten natmaken.
\eindLijst{}  

\vervolgLijst{}
    \item \EenSpeler mag geen kaarten op de grond laten vallen.
\eindLijst{}  

\vervolgLijst{}
    \item \EenSpeler mag geen schade toebrengen aan fritsgerelateerde objecten. 
\eindLijst{}


\customBoxItalic{Het is belangrijk dat je serieus meespeelt. Zo niet, moet je stoppen met spelen.}

\vervolgLijst{}
    \item Als \eenSpeler volgens minimaal twee derde van \alleSpelers niet serieus meedoet met het huidige potje Fritsen, worden de volgende handelingen in de gegeven volgorde uitgevoerd:
    \numeriekeLijst{}
        \item Één van de volgende handelingen:
        \puntLijst{}
            \item Zijn/haar kaarten worden onder de \ul{dichte} \textbf{stapel} gelegd als er een \ul{dichte} \textbf{stapel} is. 
            \item Zijn/haar kaarten worden als nieuwe \ul{dicht}e \textbf{stapel} neergelegd als er geen bestaande \ul{dichte} \textbf{stapel} is.
        \eindPuntLijst{}
        \item De desbetreffende speler stopt met Fritsen.
    \eindNumeriekeLijst{}
\eindLijst{}

\vervolgLijst{}
    \item Als \eenSpeler volgens minimaal twee derde van \alleSpelers niet serieus meedoet met het huidige potje Fritsen, mogen \alleSpelers te allen tijde zijn/haar kaarten pakken en bekijken.
\eindLijst{}

\customBoxItalic{Blijf goed opletten of andere spelers geen fouten maken. Zo niet, moet je mogelijk drinken.}

\vervolgLijst{}
    \item Als de \vorigeSpeler één van de regels van dit document onopgemerkt heeft overtreden en dit kenbaar maakt tussen het moment dat de \huidigeSpeler met zijn/haar beurt gestart is en zijn/haar beurt beëindigt heeft, moet iedereen behalve de \textbf{[vorige speler]}:
    \puntLijst{}
        \item Proosten op \quotes{supermooi Victor} en \textbf{een Victortje nemen}\footnotemark[1].
    \eindPuntLijst{}
\eindLijst{}   

\footnotetext[1]{\textbf{een Victortje nemen} is equivalent aan \textbf{Fritsen} (zie definitie \ref{item:enkel_fritsen_equivalent})}

\newpage
\drawBar{Regels - Vervolg}
\deelhoofdstuk{Regels - Beginfase van het Spel}
\label{sec:beginfase_start}

\customBoxItalic{Fritsen begint nadat \Frits en \Willem zijn bepaald. Hierna schudt iemand de kaarten en geeft \Frits iedereen vijf \ul{dichte} startkaarten om vervolgens de eerste stapel open te leggen. De regels in deze sectie beschrijven het spelverloop op pagina \pageref{sec:introductie}.}

\vervolgLijst{}
    \item Fritsen wordt met minimaal 2 en maximaal 10 spelers gespeeld.
\eindLijst{}

\vervolgLijst{}
    \item Fritsen is begonnen zodra \Frits en \Willem zijn bepaald.
    \label{regel:Willen_Frits_bepalen}
\eindLijst{}

\vervolgLijst{}
    \item \Frits moet, nadat regel \ref{regel:Willen_Frits_bepalen} toegepast is, binnen 99 seconden ervoor zorgen dat één van \alleSpelers kan \textbf{vuil Fritsen}.
\eindLijst{}

\customBoxItalic{Fritsen wordt met één of twee pakken kaarten gespeeld. De kaarten moeten goed en zichtbaar worden geschud. Het wordt gewaardeerd wanneer \Frits de kaarten schudt, echter kan \Frits iemand verplichten om te schudden.}

\vervolgLijst{}
    \item Fritsen wordt, tenzij regel \ref{regel:twee_pakken_minder_dan_7_personsen} toegepast is, tot en met 6 spelers met één \textbf{pak kaarten}\footnotemark[1] en tussen de 7 en 10 spelers met twee \textbf{pakken kaarten}\footnotemark[1] gespeeld.
\eindLijst{}

\vervolgLijst{}
    \item Fritsen wordt met twee \textbf{pakken kaarten}\footnotemark[1] gespeeld wanneer de groep van \alleSpelers minder dan 7 spelers bevat en meer dan de helft van \alleSpelers het spelen met twee \textbf{pakken kaarten}\footnotemark[1] goedkeuren. \label{regel:twee_pakken_minder_dan_7_personsen}
\eindLijst{}

\vervolgLijst{}
    \item Het \textbf{pak kaarten}\footnotemark[1] dat of de \textbf{pakken kaarten}\footnotemark[1] die gebruikt worden voor het huidige potje Fritsen moet(en) goed volgens en zichtbaar voor meer dan de helft van \alleSpelers geschud worden.
    \label{regel:goed_zichtbaar_schudden}
\eindLijst{}

\vervolgLijst{}
    \item Als \eenSpeler regel \ref{regel:goed_zichtbaar_schudden} niet naleeft, moet hij/zij in de gegeven volgorde:
    \puntLijst{}
        \item Proosten op \quotes{Frits} en \textbf{een Fritsje des nemen}\footnotemark[2].
        \item Alle kaarten verzamelen en schudden volgens regel \ref{regel:goed_zichtbaar_schudden}.
        \item Alle kaarten aan \Frits geven zodat hij/zij ze kan delen.
    \eindPuntLijst{}
\eindLijst{}

\vervolgLijst{}
    \item \Frits kan \eenSpeler verplichten om de kaarten te schudden volgens regel \ref{regel:goed_zichtbaar_schudden} door:
    \puntLijst{}
        \item Dit aan te geven en \textbf{een dubbele Frits}\footnotemark[3] te nemen.
    \eindPuntLijst{}
\eindLijst{}

\customBoxItalic{Onderstaande regels beschrijven stap 2 van het spelverloop op pagina \pageref{sec:introductie}.}

\vervolgLijst{}
    \item \Willem moet, voordat \Frits begonnen is met delen, direct links plaatsnemen van \textbf{[Frits]}.
    \label{regel:Willen_links_van_Frits}
\eindLijst{}

\vervolgLijst{}
    \item Als \Frits en \Willem regel \ref{regel:Willen_links_van_Frits} niet naleven, moeten zij:
    \puntLijst{}
        \item Proosten op \quotes{Frits} en \textbf{een Fritsje des nemen}\footnotemark[1].
        \item Zich positioneren volgens regel \ref{regel:Willen_links_van_Frits}.
    \eindPuntLijst{}
\eindLijst{}

\footnotetext[1]{zie definities \ref{item:kaarten} en \ref{item:kaarten_2}}
\footnotetext[2]{\textbf{een Fritsje des nemen} is equivalent aan \textbf{Fritsen} (zie definitie \ref{item:enkel_fritsen_equivalent})}
\footnotetext[3]{\textbf{een dubbele Frits nemen} is equivalent aan twee keer \textbf{Fritsen} (zie definities \ref{item:enkel_fritsen_equivalent} en \ref{item:dubbelfritsen_equivalent})}

\newpage
\drawBar{Regels - Vervolg}
\deelhoofdstuk{Regels - Beginfase van het spel - Vervolg}
\label{sec:beginfase_einde}

\customBoxItalic{Onderstaande regels beschrijven stap 3 en 4 van het spelverloop op pagina \pageref{sec:introductie}.}

\vervolgLijst{}
    \item Alleen \Frits mag delen\footnotemark[1].
    \label{regel:delen_Frits}
\eindLijst{}

\vervolgLijst{}
    \item \Frits is niet verplicht om tijdens het delen\footnotemark[1] de kaarten netjes, recht of gepast te positioneren.
\eindLijst{}

\vervolgLijst{}
    \item Als \Frits deelt\footnotemark[1], moet hij/zij \alleSpelers vijf \ul{dichte} kaarten geven van de \ul{dichte} \textbf{stapel} door deze één voor één van bovenop de \ul{dichte} \textbf{stapel} te pakken.
    \label{regel:delen_Frits_5_kaarten_1}
\eindLijst{}

\vervolgLijst{}
    \item Als \Frits deelt\footnotemark[1], moet hij/zij \alleSpelers om de beurt één \ul{dichte} kaart geven.
    \label{regel:delen_Frits_5_kaarten_2}
\eindLijst{}

\vervolgLijst{}
    \item Nadat \Frits regels \ref{regel:delen_Frits_5_kaarten_1} en \ref{regel:delen_Frits_5_kaarten_2} heeft toegepast, moet hij/zij direct de \ul{bovenste} kaart van de \ul{dichte} \textbf{stapel} openleggen direct naast de \ul{dichte} \textbf{stapel}.
    \label{regel:eerst_kaart_open}
\eindLijst{}   

\vervolgLijst{}
    \item Als \Frits een \kaart{joker}\footnotemark[2] openlegt in regel \ref{regel:eerst_kaart_open}, moet hij/zij nogmaals de \ul{bovenste} kaart van de \ul{dichte} \textbf{stapel} \ul{open} op de enige \ul{open} \textbf{stapel} leggen totdat deze geen \kaart{joker}\footnotemark[2] is.
    \label{regel:eerste_kaart_open_joker}
\eindLijst{}   

\vervolgLijst{}
    \item \EenSpeler mag alleen zijn/haar eigen \textbf{stapel} met \ul{dichte} kaarten pakken nadat \Frits gedeeld\footnotemark[1] heeft.
   \label{regel:andere_kaarten_pakken}
\eindLijst{}

\vervolgLijst{}
    \item Als \eenSpeler regel \ref{regel:delen_Frits}, \ref{regel:delen_Frits_5_kaarten_1}, \ref{regel:delen_Frits_5_kaarten_2}, \ref{regel:eerst_kaart_open}, \ref{regel:eerste_kaart_open_joker} of \ref{regel:andere_kaarten_pakken} niet naleeft, moet hij/zij in de gegeven volgorde:
    \puntLijst{}
        \item Proosten op \quotes{Frits} en \textbf{een Fritsje des nemen}\footnotemark[3].
        \item Alle kaarten verzamelen en schudden volgens regel \ref{regel:goed_zichtbaar_schudden}.
        \item Alle kaarten aan \Frits geven zodat hij/zij ze opnieuw kan delen.
    \eindPuntLijst{}
\eindLijst{}

\vervolgLijst{}
    \item Als \eenSpeler kaarten in zijn/haar hand neemt terwijl \Frits deelt\footnotemark[1], moet hij/zij:
     \puntLijst{}
         \item Proosten op \quotes{Frits} en \textbf{een Fritsje des nemen}\footnotemark[3].
         \item Zijn/haar gepakte kaart(en) in zijn/haar hand houden.
     \eindPuntLijst{}
\eindLijst{}

\customBoxItalic{Onderstaande regels beschrijven stap 6 van het spelverloop op pagina \pageref{sec:introductie}.}

\vervolgLijst{}
    \item Tijdens het Fritsen, nadat \Willem zijn/haar eerste kaart(en) heeft opgelegd, zijn\\ \alleSpelers in de richting van de klok aan de beurt\footnotemark[4].
\eindLijst{}


\footnotetext[1]{zie \texttt{stap 3} op pagina \pageref{sec:introductie}}
\footnotetext[2]{zie definitie \ref{item:kaarten}}
\footnotetext[3]{\textbf{een Fritsje des nemen} is equivalent aan \textbf{Fritsen} (zie definitie \ref{item:enkel_fritsen_equivalent})}
\footnotetext[4]{zie het stroomdiagram op de achterkant van dit document}

\newpage
\drawBar{Regels - Vervolg}
\deelhoofdstuk{Regels - Vuil Fritsen}
\label{sec:vuil_fritsen}

\customBoxItalic{\textbf{Vuil Fritsen} is het drinken en daarna inwisselen van je vijf startkaarten voordat \Willem als eerste aan de beurt\footnotemark[1] is. Een slechte hand bestaat uit meerdere kaarten met de waarde lager dan een \ul{\textbf{9}}. Je mag zo vaak \textbf{vuil Fritsen} als je wilt.}

\vervolgLijst{}
    \item Als \eenSpeler aan het \textbf{vuil Fritsen} is, moet hij/zij in de gegeven volgorde:
    \puntLijst{}
        \item Proosten op \quotes{vuile Frits} en \textbf{een dubbele Frits nemen}\footnotemark[3].
        \item Zijn/haar kaarten \ul{dicht} op de snijplank leggen.
        \item Vijf nieuwe \ul{dichte} kaarten van \Frits in ontvangst nemen van bovenop de \ul{dichte} \textbf{stapel} nadat \Frits de \ul{oude} kaarten \ul{dicht} onderop de \ul{dichte} \textbf{stapel} gelegd heeft\footnotemark[2].
    \eindPuntLijst{}
     \label{regel:zet_vuil_fritsen}
\eindLijst{}

\vervolgLijst{}
    \item Als minimaal twee derde van \alleSpelers bepaalt dat, voordat regel \ref{regel:zet_vuil_fritsen} toegepast is en voordat \'e\'en van \alleSpelers zijn/haar kaarten gezien heeft, géén van \alleSpelers mag \textbf{vuil Fritsen}, mag géén van \alleSpelers \textbf{vuil Fritsen}.
    \label{regel:skip_vuil_fritsen}
\eindLijst{}

\vervolgLijst{}
    \item Als \eenSpeler ongelijk aan \Frits kaarten van de \ul{dichte} \textbf{stapel} pakt tijdens het uitvoeren van zet \ref{regel:zet_vuil_fritsen}, moet hij/zij:
    \puntLijst{}
        \item Proosten op \quotes{Frits} en \textbf{een Fritsje des nemen}\footnotemark[4].
        \item Zijn/haar gepakte kaart(en) in zijn/haar hand houden.
    \eindPuntLijst{}
\eindLijst{}

\vervolgLijst{}
    \item \EenSpeler mag alleen \textbf{vuil Fritsen} als \ul{g\'e\'en} van \textbf{[alle spelers]} aan de beurt\footnotemark[1] geweest is en regel \ref{regel:skip_vuil_fritsen} niet toegepast is. 
\eindLijst{}

\vervolgLijst{}
    \item \EenSpeler mag, tenzij regel \ref{regel:skip_vuil_fritsen} toegepast is, zo vaak \textbf{vuil Fritsen} als hij/zij wil. 
\eindLijst{}

\vervolgLijst{}
    \item \EenSpeler mag alleen proosten op \quotes{vuile Frits} als hij/zij aan het \textbf{vuil Fritsen} is.
\eindLijst{}

\vervolgLijst{}
    \item Bij \textbf{vuil Fritsen} geldt wie het eerst komt, het eerst maalt.
\eindLijst{}

\customBoxItalic{Je hebt genoeg tijd om te beslissen of je wilt \textbf{vuil Fritsen}.}

\vervolgLijst{}
    \item \Willem mag de eerste 99 seconden na het uitdelen van de kaarten alleen met zijn/haar beurt beginnen wanneer \textbf{[alle spelers]} dit goedkeuren of regel \ref{regel:skip_vuil_fritsen} toegepast is. 
    \label{regel:vuile_frits_1}
\eindLijst{}

\vervolgLijst{}
    \item \Willem mag pas na 99 seconden na de laatste \textbf{vuile Frits} met zijn/haar beurt beginnen of wanneer \textbf{[alle spelers]} het goedkeuren dat hij/zij eerder begint.
    \label{regel:vuile_frits_2}
\eindLijst{}

\vervolgLijst{}
    \item Als \Willem regel \ref{regel:vuile_frits_1} of \ref{regel:vuile_frits_2} niet naleeft, moet hij/zij:
    \puntLijst{}
        \item Proosten op \quotes{Frits} en \textbf{een Fritsje des nemen}\footnotemark[3].
        \item Zijn/haar beurt\footnotemark[1] meteen beëindigen.
        \item Zijn/haar opgelegde kaart(en) terug in zijn/haar hand nemen.
    \eindPuntLijst{}
    \label{regel:kaarten_terugnemen_1}
\eindLijst{}

\footnotetext[1]{zie het stroomdiagram op de achterkant van dit document}
\footnotetext[2]{zie regel \ref{regel:speler_blokkeren}}
\footnotetext[3]{\textbf{een dubbele Frits nemen} is equivalent aan twee keer \textbf{Fritsen} (zie definities \ref{item:enkel_fritsen_equivalent} en \ref{item:dubbelfritsen_equivalent})}
\footnotetext[4]{\textbf{een Fritsje des nemen} is equivalent aan \textbf{Fritsen} (zie definitie \ref{item:enkel_fritsen_equivalent})}

\newpage
\drawBar{Regels - Vervolg}
\deelhoofdstuk{Regels - Beurten en Zetten}
\label{sec:beurten_en_zetten_start}

\customBoxItalic{Tijdens Fritsen is iedereen met de klok mee aan de beurt. Zolang je nog kaarten hebt, leg je elke beurt \'e\'en of meer kaarten weg en bestaat de mogelijkheid dat je twee nieuwe kaarten moet pakken. Het stroomdiagram op achterkant van dit document beschrijft hoe een beurt verloopt. Pagina \pageref{sec:regels_kort}, en hoofdstuk 5 in detail, bevatten een overzicht van de mogelijke zetten en de bijbehorende handelingen die moeten worden uitgevoerd.}

\vervolgLijst{}
    \item Als \eenSpeler langer dan 99 seconden over zijn/haar beurt\footnotemark[1] doet, moet hij/zij:
    \puntLijst{}
        \item Proosten op \quotes{Frits} en \textbf{een Fritsje des nemen}\footnotemark[2]
        \item Twee \ul{dichte} kaarten pakken van bovenop de \ul{dichte} \textbf{stapel} wanneer hij/zij dit nog niet gedaan heeft in zijn/haar beurt.
        \item Zijn/haar beurt\footnotemark[1] meteen beëindigen.
    \eindPuntLijst{}
    \label{regel:beurt_langer_dan_99}
\eindLijst{}

\customBoxItalic{Onderstaande regels beschrijven het stroomdiagram op de achterkant van dit document.}

\vervolgLijst{}
    \item \EenSpeler mag meteen na het beginnen van zijn/haar beurt\footnotemark[1] en voordat hij/zij kaart(en) gepakt heeft \'e\'en zet\footnotemark[3] doen.
    \label{regel:zet_zonder_pakker}
\eindLijst{}

\vervolgLijst{}
    \item \label{regel:twee_kaarten} Als de \huidigeSpeler geen zet\footnotemark[3] doet voordat hij/zij twee nieuwe \ul{dichte} kaarten heeft gepakt, moet hij/zij in de gegeven volgorde:
    \puntLijst{}
        \item \textbf{Fritsen}
        \item Twee nieuwe \ul{dichte} kaarten pakken van bovenop de \ul{dichte} \textbf{stapel}.
        \item Één van de volgende handelingen uitvoeren:
        \numeriekeLijst{}
            \item Een zet\footnotemark[3] doen.
            \item Een nieuwe \ul{open} \textbf{stapel} beginnen.
        \eindNumeriekeLijst{}
     \eindPuntLijst{}
\eindLijst{}

\vervolgLijst{}
    \item Proosten op \quotes{Frits} tijdens het toepassen van regel \ref{regel:twee_kaarten} is niet verplicht maar wordt wel gewaardeerd.
\eindLijst{}

\vervolgLijst{}
    \item Als \eenSpeler twee \ul{dichte} kaarten moet pakken tijdens het toepassen van regel \ref{regel:kijken_in_dichte_stapel}, \ref{regel:beurt_langer_dan_99} of \ref{regel:twee_kaarten} terwijl er g\'e\'en \ul{dichte} kaarten meer zijn, hoeft hij/zij geen \ul{dichte} kaarten te pakken.
    \label{item:geen_kaart_1}
\eindLijst{}

\vervolgLijst{}
    \item Als \eenSpeler twee \ul{dichte} kaarten moet pakken tijdens het toepassen van regel \ref{regel:kijken_in_dichte_stapel}, \ref{regel:beurt_langer_dan_99} of \ref{regel:twee_kaarten} terwijl er nog maar \'e\'en \ul{dichte} kaart is, moet hij/zij \'e\'en \ul{dichte} kaart pakken.
    \label{item:geen_kaart_2}
\eindLijst{}

\vervolgLijst{}
    \item Als \eenSpeler méér kaart(en) dan nodig pakt van de \ul{dichte} \textbf{stapel}, moet hij/zij deze in zijn/haar hand houden.
\eindLijst{}

\vervolgLijst{}
    \item \EenSpeler is niet verplicht om een zet\footnotemark[3] te doen voordat en nadat hij/zij twee nieuwe kaarten heeft gepakt.
\eindLijst{}

\vervolgLijst{}
    \item \EenSpeler moet in zijn/haar beurt\footnotemark[1] \'e\'en zet\footnotemark[3] doen of \'e\'en nieuwe \ul{open} \textbf{stapel} beginnen als hij/zij nog kaarten in zijn/haar hand heeft.
\eindLijst{}

\footnotetext[1]{zie het stroomdiagram op de achterkant van dit document}
\footnotetext[2]{\textbf{een Fritsje des nemen} is equivalent aan \textbf{Fritsen} (zie definitie \ref{item:enkel_fritsen_equivalent})}
\footnotetext[3]{zie pagina \pageref{sec:regels_kort}, \pageref{sec:zettenLang} en \pageref{sec:zettenLang_2}}

\newpage
\drawBar{Regels - Vervolg}
\deelhoofdstuk{Regels - Beurten en Zetten - Vervolg}

\customBoxItalic{Leg je kaarten netjes neer.}

\vervolgLijst{}
    \item Als \eenSpeler een kaart neerlegt, moet hij/zij deze netjes, recht en gepast positioneren volgens meer dan de helft van \textbf{[alle spelers]}.
    \label{regel:kaart_netjes_1}
\eindLijst{}

\vervolgLijst{}
    \item Als \eenSpeler één van de opgelegde kaart(en) of één van de kaarten van de \ul{dichte} \textbf{stapel} verplaatst, moet hij/zij deze netjes, recht en gepast positioneren volgens meer dan de helft van \textbf{[alle spelers]}.
    \label{regel:kaart_netjes_2} 
\eindLijst{}

\vervolgLijst{}
    \item Als \eenSpeler regel \ref{regel:kaart_netjes_1} of \ref{regel:kaart_netjes_2} niet naleeft, moet hij/zij:
    \puntLijst{}
        \item Proosten op \quotes{Frits} en \textbf{een Fritsje des nemen}\footnotemark[1].
        \item Alle \ul{open} \textbf{stapel(s)} en de \ul{dichte} \textbf{stapel} recht, netjes en gepast positioneren volgens meer dan de helft van \textbf{[alle spelers]}. 
    \eindPuntLijst{}
\eindLijst{}

\customBoxItalic{Proost alleen op het juiste moment.}

\vervolgLijst{}
    \item \EenSpeler mag alleen proosten op \quotes{offer Frits}, \quotes{kut Kim}, \quotes{Chantal}, \quotes{kut Lisa}, \quotes{Baudet}, \quotes{supermooi Erik} en \quotes{supermooi Victor} als hij/zij de bijbehorende zet aan het uitvoeren of regel aan het toepassen is\footnotemark[2]. 
\eindLijst{}

\customBoxItalic{Leg alleen een kaart op wanneer dit mag.}

\vervolgLijst{}
    \item \EenSpeler mag alleen kaart(en) opleggen die in lijn zijn met de regels en zetten.
    \label{regel:ongeldige_zet}
\eindLijst{}

\vervolgLijst{}
    \item \EenSpeler mag, tenzij elders beschreven, niet voor zijn/haar beurt\footnotemark[3] opleggen.
    \label{regel:voor_beurt_opleggen}
\eindLijst{}

\vervolgLijst{}
    \item \EenSpeler mag geen zet doen wanneer het spel is gepauzeerd\footnotemark[4].
    \label{regel:tijdens_pauze_opleggen}
\eindLijst{}

\vervolgLijst{}
    \item Als \eenSpeler regel \ref{regel:ongeldige_zet}, \ref{regel:voor_beurt_opleggen} of \ref{regel:tijdens_pauze_opleggen} niet naleeft, moet hij/zij:
    \puntLijst{}
        \item Proosten op \quotes{Frits} en \textbf{een Fritsje des nemen}\footnotemark[1].
        \item De opgelegde kaart(en) terug in zijn/haar hand nemen.
    \eindPuntLijst{}
    \label{regel:kaarten_terugnemen_2}
\eindLijst{}

\vervolgLijst{}
    \item Als \eenSpeler één kaart of meerdere kaarten oplegt die niet in lijn zijn met de regels en zetten, heeft deze kaart of hebben deze kaarten geen drinkwaarde.
\eindLijst{}  

\footnotetext[1]{\textbf{een Fritsje des nemen} is equivalent aan \textbf{Fritsen} (zie definitie \ref{item:enkel_fritsen_equivalent})}
\footnotetext[2]{zie zetten \ref{zet:offer_frits}, \ref{zet:lisa} en \ref{zet:kim} op pagina \pageref{zet:kim}, zetten \ref{zet:joris} en  \ref{zet:thierry} op pagina \pageref{zet:joris} en regel \ref{zet:Erik}}
\footnotetext[3]{zie het stroomdiagram op de achterkant van dit document}
\footnotetext[4]{zie regels \ref{regel:stilleggen_1}, \ref{regel:stilleggen_2}, \ref{regel:stilleggen_3} en \ref{regel:stilleggen_4}}

\newpage
\drawBar{Regels - Vervolg}
\deelhoofdstuk{Regels - Beurten en Zetten - Vervolg}
\label{sec:beurten_en_zetten_einde}

\customBoxItalic{Blijf goed opletten wie er aan de beurt\footnotemark[1] is of wie geen kaarten meer heeft.}

\vervolgLijst{}
    \item \EenSpeler moet van \alleSpelers weten wie aan de beurt\footnotemark[1] is en wie \textbf{uitgefritst}\footnotemark[2] zijn.
\eindLijst{}

\vervolgLijst{}
    \item Als niemand van \alleSpelers weet of wil vertellen wie aan de beurt\footnotemark[1] is, is \Frits aan de beurt\footnotemark[1].
\eindLijst{}

\vervolgLijst{}
    \item \EenSpeler kan zijn/haar \medeSpelers dwingen om naar waarheid te vertellen wie aan de beurt\footnotemark[1] is en wie \textbf{uitgefritst}\footnotemark[2] zijn door:
    \puntLijst{}
        \item Dit aan te geven en \textbf{een Fritsje te nemen}\footnotemark[3].
    \eindPuntLijst{}
\eindLijst{}

\customBoxItalic{Vergeet niet te proosten op \quotes{laatste Frits} (laatste kaart) en de juiste handelingen uit te voeren wanneer je uitgefritst bent of als laatste nog kaarten in je hand hebt.}

\vervolgLijst{}
    \item Als \eenSpeler na het opleggen nog \'e\'en kaart in zijn/haar hand heeft, moet hij/zij:
    \puntLijst{}
        \item Hoor- en verstaanbaar \quotes{laatste Frits} voor minimaal twee derde van \\ \textbf{[alle spelers]} zeggen.
    \eindPuntLijst{}
    \label{regel:laatste_frits_1}
\eindLijst{}

\vervolgLijst{}
    \item Als de \huidigeSpeler na het opleggen van zijn/haar kaart(en) \textbf{uitgefritst}\footnotemark[4] is, moet hij/zij:
    \puntLijst{}
        \item Proosten op \quotes{Frits} en \textbf{een uittreefritsje nemen}\footnotemark[3].
    \eindPuntLijst{}
    \label{regel:laatste_frits_2}
\eindLijst{}

\vervolgLijst{}
    \item Als \eenSpeler als laatste van \alleSpelers nog niet \textbf{uitgefritst}\footnotemark[2] is, moet hij/zij:
    \puntLijst{}
        \item Proosten op \quotes{dubbel Frits} en \textbf{een dubbele Frits nemen}\footnotemark[4].
    \eindPuntLijst{}
\eindLijst{}

\customBoxItalic{Onderstaande regels beschrijven wat er gebeurt als iemands beurt ten einde is en wanneer het spel is afgelopen.}

\vervolgLijst{}
    \item Een beurt\footnotemark[1] van de \huidigeSpeler is ten einde nadat hij/zij  regel \ref{regel:beurt_langer_dan_99}, \ref{regel:zet_zonder_pakker} of \ref{regel:twee_kaarten} heeft uitgevoerd en al zijn/haar beurtgerelateerde taken heeft verricht.
\eindLijst{}

\vervolgLijst{}
    \item De beurt\footnotemark[1] van \eenSpeler is meteen ten einde wanneer hij/zij \textbf{uitgefritst}\footnotemark[2] is aan het begin van zijn/haar beurt. 
\eindLijst{}

\vervolgLijst{}
    \item Als er nog maar \textbf{[\'e\'en speler]} niet \textbf{uitgefritst}\footnotemark[2] is en niemand van \alleSpelers meer hoeft te drinken, is het spel afgelopen.
\eindLijst{}

\vervolgLijst{}
    \item \textbf{[Alle spelers]} moeten blijven meespelen totdat nog maar \textbf{[één speler]} niet \textbf{uitgefritst}\footnotemark[3] is en hij/zij al zijn/haar beurtgerelateerde taken heeft verricht.
\eindLijst{}  

\vervolgLijst{}
    \item \textbf{[Alle spelers]} moeten blijven meespelen totdat niemand van \alleSpelers meer hoeft te drinken.
\eindLijst{} 

\footnotetext[1]{zie het stroomdiagram op de achterkant van dit document}
\footnotetext[2]{\textbf{uitfritsen} is het neerleggen van je laatste kaart(en) (zie definitie \ref{item:uitfritsen})}
\footnotetext[3]{\textbf{een Fritsje des nemen} en \textbf{een uittreefritsje nemen} zijn equivalent aan \textbf{Fritsen} (zie definitie \ref{item:enkel_fritsen_equivalent})}
\footnotetext[4]{\textbf{een dubbele Frits nemen} is equivalent aan twee keer \textbf{Fritsen} (zie definities \ref{item:enkel_fritsen_equivalent} en \ref{item:dubbelfritsen_equivalent})}



\newpage
\drawBar{Regels - Vervolg}
\deelhoofdstuk{Regels - Jokers}
\label{sec:jokers}

\customBoxItalic{Tijdens je beurt\footnotemark[1] kun je andere spelers laten drinken door een \kaart{joker}\footnotemark[2] op te leggen. Deze kaart heeft zijn eigen stapel en je mag hem niet als laatste neerleggen.}

\vervolgLijst{}
    \item Er moeten 2 of 3 \kaart{jokers}\footnotemark[2] per kaartspel gebruikt worden.
\eindLijst{}

\vervolgLijst{}
    \item \EenSpeler mag niet \textbf{uitfritsen}\footnotemark[3] met een \kaart{joker}\footnotemark[2].
    \label{regel:joker_1}
\eindLijst{}

\vervolgLijst{}
    \item \EenSpeler mag niet een nieuwe \textbf{jokerstapel} beginnen als er al \'e\'en ligt.
    \label{regel:joker_2}
\eindLijst{}

\vervolgLijst{}
    \item \EenSpeler mag, tenzij elders beschreven, niet een kaart ongelijk aan een \kaart{joker}\footnotemark[2] op de \textbf{jokerstapel} leggen.
    \label{regel:joker_3}
\eindLijst{}

\vervolgLijst{}
    \item Als \eenSpeler een \kaart{joker}\footnotemark[2] neerlegt moet deze zich direct naast de \ul{dichte} \textbf{stapel} bevinden.
    \label{regel:plaats_joker}
\eindLijst{}

\vervolgLijst{}
    \item \EenSpeler mag niet een kaart opleggen waardoor de \textbf{jokerstapel} zich niet direct naast de \ul{dichte} \textbf{stapel} kan bevinden.
    \label{regel:joker_4}
\eindLijst{}

\vervolgLijst{}
    \item Als \eenSpeler regel \ref{regel:joker_1}, \ref{regel:joker_2}, \ref{regel:joker_3}, \ref{regel:plaats_joker} of \ref{regel:joker_4} niet naleeft, moet hij/zij:
    \puntLijst{}
        \item Proosten op \quotes{Frits} en \textbf{een Fritsje des nemen}\footnotemark[4].
        \item De opgelegde kaart(en) terug in zijn/haar hand nemen.
    \eindPuntLijst{}
    \label{regel:kaarten_terugnemen_3}
\eindLijst{}

\customBoxItalic{Nadat \Frits iedereen 5 dichte startkaarten gegeven heeft, legt \Frits de eerste stapel open. In het geval van een joker, legt \Frits een nieuwe kaart eroverheen en herhaalt dit tot deze nieuwe kaart geen joker meer is. In deze situatie hoeft niemand te drinken.}

\vervolgLijst{}
    \item Als \Frits regel \ref{regel:eerst_kaart_open} of \ref{regel:eerste_kaart_open_joker} aan het toepassen is, hebben de opgelegde \kaart{joker(s)}\footnotemark[2] geen drinkwaarde en worden ze niet gezien als een \textbf{jokerstapel}.
\eindLijst{}   

\footnotetext[1]{zie het stroomdiagram op de achterkant van dit document}
\footnotetext[2]{zie definities \ref{item:kaarten} en \ref{item:kaarten_2}}
\footnotetext[3]{\textbf{uitfritsen} is het neerleggen van je laatste kaart(en) (zie definitie \ref{item:uitfritsen})}
\footnotetext[4]{\textbf{een Fritsje des nemen} is equivalent aan \textbf{Fritsen} (zie definitie \ref{item:enkel_fritsen_equivalent})}

\newpage
\drawBar{Regels - Vervolg}
\deelhoofdstuk{Regels - Thierry Baudet en Jesse Klaver}
\label{sec:thierry}

\customBox{\textit{Zet }\ref{zet:thierry}\footnotemark[1] (Thierry)\textit{ is het omruilen van je kaarten midden in het potje Fritsen nadat je een \kaart{6} op een \kaart{vrouw} hebt gelegd, hebt geproost op \quotes{Baudet} en dubbel hebt gedronken. Thierry Baudet wil het partijkartel omverwerpen. Je weggelegde kaarten zijn het partijkartel en de nieuwe gevestigde orde is de nieuwe set kaarten die je krijgt van \textbf{[Frits]}.}}

\vervolgLijst{}
    \item Als \eenSpeler mag niet \textbf{uitfritsen} met zet \ref{zet:thierry}\footnotemark[1].
    \label{regel:thierry_1}
\eindLijst{}

\vervolgLijst{}
    \item Als \eenSpeler regel \ref{regel:thierry_1} niet naleeft, moet hij/zij:
    \puntLijst{}
        \item Proosten op \quotes{Frits} en \textbf{een Fritsje des nemen}\footnotemark[2].
        \item De opgelegde kaart terug in zijn/haar hand nemen.
    \eindPuntLijst{}
    \label{regel:kaarten_terugnemen_4}
\eindLijst{}

\customBox{\textit{Jesse Klaver wil niet dat Thierry Baudet de macht krijgt. Als je zet }\ref{zet:thierry}\footnotemark[1]\textit{ aan het uitvoeren bent, kan een anders speler het inruilen van jouw hand voorkomen door direct na het opleggen van jouw \kaart{6} een \kaart{klaveren 3} eroverheen te leggen.}}

\vervolgLijst{}
    \item \label{zet:Jesse} Tussen het moment dat de \huidigeSpeler zet \ref{zet:thierry}\footnotemark[1] heeft gestart en nieuwe kaarten heeft gekregen van \textbf{[Frits]}, mag \eenSpeler een \kaart{Jesse Klaver}\footnotemark[3] direct op de net opgelegde \kaart{Thierry}\footnotemark[4] leggen.
 \eindLijst{}
  
\vervolgLijst{}   
    \item \label{zet:Jesse_2} Als \eenSpeler regel \ref{zet:Jesse} heeft toegepast, moet hij/zij:
    \puntLijst{}
        \item Zijn/haar weggelegde kaarten terug in zijn/haar hand nemen.
        \item Stoppen met het inwisselen van zijn/haar kaarten. 
    \eindPuntLijst{}
\eindLijst{}

\vervolgLijst{}
    \item \Frits mag pas 9 seconden nadat \eenSpeler zet \ref{zet:thierry}\footnotemark[1] heeft gestart zijn/haar kaarten inwisselen. 
\eindLijst{}

\vervolgLijst{}
    \item \Frits mag geen kaarten bekijken die hij/zij voor een \andereSpeler aan het inwisselen is.
\eindLijst{}

\vervolgLijst{}
    \item \label{zet:Erik} Als \eenSpeler in dezelfde beurt volgens regel \ref{zet:Jesse} een \kaart{Jesse Klaver}\footnotemark[3] op een \kaart{Thierry}\footnotemark[4] legt, moet hij/zij:
    \puntLijst{}
        \item Proosten op \quotes{supermooi Erik} en \textbf{een Erikje nemen}\footnotemark[6].
    \eindPuntLijst{}
\eindLijst{}

\vervolgLijst{}
    \item \label{zet:Erik_2} \EenSpeler moet, tenzij hij/zij regel \ref{zet:Erik} aan het toepassen is, te allen tijde proosten op \quotes{Baudet} en \textbf{een Thierry'tje nemen}\footnotemark[6] bij het toepassen van zet \ref{zet:thierry}\footnotemark[1].
\eindLijst{}

\footnotetext[1]{zie pagina \pageref{zet:thierry}}
\footnotetext[2]{\textbf{een Fritsje des nemen} is equivalent aan \textbf{Fritsen} (zie definitie \ref{item:enkel_fritsen_equivalent})}
\footnotetext[3]{de kaart \kaart{Jesse Klaver} is equivalent aan een \kaart{klaveren 3} (zie pagina \pageref{sec:kaartnamen})}
\footnotetext[4]{de kaart \kaart{Thierry} is equivalent aan elke \kaart{6} (zie pagina \pageref{sec:kaartnamen})}
\footnotetext[5]{zie het stroomdiagram op de achterkant van dit document}
\footnotetext[6]{\textbf{een Erikje nemen} en \textbf{een Thierry'tje nemen} zijn equivalent aan \textbf{Dubbelfritsen} (zie definitie \ref{item:dubbelfritsen_equivalent})}
