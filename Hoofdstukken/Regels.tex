\newpage
\drawBar{}

\hoofdstuk{2}{Regels}

\begin{tabular}{llll}
    \large{$\rightarrow$ Algemeen}                       & \hspace{0.5cm} & \vspace{0.10cm}    & \hspace{0.25cm} \large{pagina \pageref{sec:algemeen_start} t/m \pageref{sec:algemeen_einde}}                   \\
    \large{$\rightarrow$ Beginfase van het Spel}         & \hspace{0.5cm} & \vspace{0.10cm}    & \hspace{0.25cm} \large{pagina \pageref{sec:beginfase_start} t/m \pageref{sec:beginfase_einde}}                 \\
    \large{$\rightarrow$ Vuil Fritsen}                   & \hspace{0.5cm} & \vspace{0.10cm}    & \hspace{0.25cm} \large{pagina \pageref{sec:vuil_fritsen}}                                                      \\
    \large{$\rightarrow$ Beurten en Zetten}              & \hspace{0.5cm} & \vspace{0.10cm}    & \hspace{0.25cm} \large{pagina \pageref{sec:beurten_en_zetten_start} t/m \pageref{sec:beurten_en_zetten_einde}} \\
    \large{$\rightarrow$ Jokers}                         & \hspace{0.5cm} & \vspace{0cm}       & \hspace{0.25cm} \large{pagina \pageref{sec:jokers}}                                                            \\ \vspace{0cm}
    \large{$\rightarrow$ Thierry Baudet en Jesse Klaver} & \hspace{0.5cm} & \proLabelMetRuimte & \hspace{0.25cm} \large{pagina \pageref{sec:thierry}}                                                           \\
    \large{$\rightarrow$ Caroline van der Plas}          & \hspace{0.5cm} & \proLabelMetRuimte & \hspace{0.25cm} \large{pagina \pageref{regel:caroline_uitfritsen} t/m \pageref{regel:shotglaasje_aanraken_1}}
\end{tabular}

\vspace{0.5cm}
\deelhoofdstuk{Regels - Algemeen}
\label{sec:algemeen_start}
\beginLijst{2}
\item Als \eenSpelerN, tenzij elders beschreven, een regel in dit document overtreedt, moet diegene:
\puntLijst{}
\item Proosten op \quotes{Frits} en \textbf{een Fritsje des nemen}\footnotemark[1].
\eindPuntLijst{}
\eindLijst{}

\vervolgLijst{}
\item Als \eenSpeler het verboden woord \ul{\texttt{ADTEN}} zegt in welke vorm dan ook, moet diegene:
\puntLijst{}
\item Proosten op \quotes{Frits} en \textbf{een Fritsje des nemen}\footnotemark[1].
\eindPuntLijst{}
\eindLijst{}

\vervolgLijst{}
\item Als een regel binnen de huidige situatie onduidelijk is of op meerdere manieren geïnterpreteerd kan worden, moet meer dan de helft van \alleSpelers overeenkomen tot een interpretatie van de desbetreffende regel.
\eindLijst{}

\vervolgLijst{}
\item \label{regel:beslissingCriteria} Als meer dan de helft van \alleSpelers het niet eens kan worden over een fritsgerelateerde kwestie gelden de volgende beslissingscriteria (in volgorde van belangrijkheid):
\numeriekeLijst{}
\item De beslissing wordt genomen door de grootste groep van \alleSpelers met dezelfde mening over de gerelateerde kwestie.
\item De beslissing wordt genomen door \Frits mits diegene dit wilt.
\item De beslissing wordt genomen door \Willem mits diegene dit wilt.
\item De beslissing wordt genomen door de oudste van \alleSpelersN.
\eindNumeriekeLijst{}
\eindLijst{}

\vervolgLijst{}
\item Als \eenSpeler volgens meer dan de helft van \alleSpelers een domme of onnodige fritsgerelateerde fout maakt, moet diegene:
\puntLijst{}
\item Proosten op \quotes{Frits} en \textbf{een Fritsje des nemen}\footnotemark[1].
\eindPuntLijst{}
\eindLijst{}

\vervolgLijst{}
\item Voor elke handeling, gebrek aan een handeling of verkeerde uitspraak mag \eenSpeler maximaal twee keer opgelegd worden om te drinken.
\eindLijst{}

\vervolgLijst{}
\item \label{regel:niet_hoeven_drinken} Als meer dan de helft van \alleSpelers goedkeurt dat \eenSpeler niet hoeft te drinken, hoeft diegene niet te drinken.
\eindLijst{}

\vervolgLijst{}
\item Het meer \textbf{Fritsen} dan nodig wordt niet beloond maar wel gewaardeerd.
\eindLijst{}

\vervolgLijst{}
\item Als \eenSpeler niet bezig is met een fritsgerelateerde taak, wordt het gewaardeerd als diegene lege shotglaasjes vult met de drank die op dat moment gebruikt wordt om te Fritsen.
\eindLijst{}

\vervolgLijst{}
\item Als \eenSpeler proost bij het uitvoeren van een fritsgerelateerde taak, moet diegene dit hoor- en verstaanbaar voor minimaal twee derde van \alleSpelers doen.
\eindLijst{}

\vervolgLijst{}
\item \EenSpeler hoeft niet de waarheid te spreken als diegene zegt \textbf{uitfritsgarantie}\footnotemark[2] te hebben.
\eindLijst{}

\vervolgLijst{}
\item \EenSpeler mag \medeSpelersN, tenzij elders beschreven, niet bewust blokkeren.
\label{regel:speler_blokkeren}
\eindLijst{}

\footnotetext[1]{\textbf{een Fritsje des nemen} is equivalent aan \textbf{Fritsen} (zie definitie \ref{item:enkel_fritsen_equivalent} op pagina \pageref{item:enkel_fritsen_equivalent})}
\footnotetext[2]{\textbf{uitfritsgarantie} hebben geen \textbf{kaarten} meer te hoeven pakken (zie definitie \ref{item:uitfritsgarantie} op pagina \pageref{item:uitfritsgarantie})}

\newpage
\drawBar{Regels - Vervolg}
\deelhoofdstuk{Regels - Algemeen - Vervolg}

\customBoxItalic{Je kunt gewoon naar het toilet en bevorderende fritsgerelateerde handelingen uitvoeren. In deze gevallen kan het spel worden gepauzeerd: niemand mag dan een zet doen.}

\vervolgLijst{}
\item Het spel wordt gepauzeerd als \eenSpeler hoor- en verstaanbaar voor minimaal twee derde van \alleSpelers duidelijk maakt naar het toilet te willen en daarna binnen 9 seconden daadwerkelijk zich naar het toilet begeeft.
\label{regel:stilleggen_1}
\eindLijst{}

\vervolgLijst{}
\item Het spel wordt gepauzeerd als \eenSpeler hoor- en verstaanbaar voor minimaal twee derde van \alleSpelers duidelijk maakt drank te willen halen die op dat moment gebruikt wordt om te Fritsen en daarna de desbetreffende drank binnen 9 seconden daadwerkelijk gaat halen.
\label{regel:stilleggen_2}
\eindLijst{}

\vervolgLijst{}
\item Het spel wordt gepauzeerd als \eenSpeler een handeling wil en gaat uitvoeren die het huidige potje Fritsen bevordert die door meer dan helft van \alleSpelers wordt goedgekeurd.
\label{regel:stilleggen_3}
\eindLijst{}

\vervolgLijst{}
\item Het spel wordt hervat wanneer \alleSpelers terug zijn na het uitvoeren van de handelingen beschreven in regels \ref{regel:stilleggen_1}, \ref{regel:stilleggen_2} en \ref{regel:stilleggen_3}.
\label{regel:stilleggen_4}
\eindLijst{}

\customBoxItalic{Pak alleen kaarten van de dichte stapel en géén kaarten van de open stapels.}

\vervolgLijst{}
\item Als \eenSpeler in de \ul{dichte} \textbf{stapel} kijkt of de \ul{dichte} \textbf{stapel} schudt, moet diegene in de gegeven volgorde:
\puntLijst{}
\item Proosten op \quotes{dubbelfrits} en een \textbf{een dubbele Frits nemen}\footnotemark[1].
\item De \ul{dichte} \textbf{stapel} goed en zichtbaar schudden volgens meer dan de helft van \\ \alleSpelersN.
\item De \ul{dichte} \textbf{stapel} op de plek terugleggen waar deze lag voor het schudden.
\eindPuntLijst{}
\label{regel:kijken_in_dichte_stapel}
\eindLijst{}

\vervolgLijst{}
\item Als \eenSpeler één \textbf{kaart} of meerdere \textbf{kaarten} van één van de \ul{open} \textbf{stapels} pakt en niet regel \ref{regel:kaarten_terugnemen_1}, \ref{regel:kaarten_terugnemen_2}, \ref{regel:kaarten_terugnemen_3}, \ref{regel:kaarten_terugnemen_4}, \ref{regel:kaarten_terugnemen_5}, \ref{regel:kaarten_terugnemen_6} of \ref{regel:kaarten_terugnemen_7} aan het toepassen is, moet diegene:
\puntLijst{}
\item Proosten op \quotes{Frits} en \textbf{een Fritsje des nemen}\footnotemark[2].
\item De net gepakte \textbf{kaart(en)} in dezelfde volgorde \ul{open} terugleggen op de desbetreffende \ul{open} \textbf{stapel}.
\eindPuntLijst{}
\eindLijst{}

\customBoxItalic{Laat niemand in je kaarten kijken en, wanneer gevraagd, vertel hoeveel je er hebt.}

\vervolgLijst{}
\item \EenSpeler mag te allen tijde in de \textbf{kaarten} van \alleSpelers kijken.
\eindLijst{}

\vervolgLijst{}
\item \EenSpeler mag niet \'e\'en of meerdere \textbf{kaarten} aanraken van een \medeSpeler die bezig is met een fritsgerelateerde taak of die door een \medeSpeler vastgehouden worden.
\eindLijst{}

\vervolgLijst{}
\item Als \eenSpeler \textbf{kaarten} vast heeft van een \medeSpelerN, moet diegene deze binnen 9 seconden teruggeven.
\eindLijst{}

\vervolgLijst{}
\item \EenSpeler is verplicht om precies en naar waarheid te vertellen hoeveel resterende \textbf{kaarten} diegene heeft wanneer dit gevraagd wordt door een \medeSpelerN.
\eindLijst{}

\footnotetext[1]{\textbf{een dubbele Frits nemen} is equivalent aan twee keer \textbf{Fritsen} (zie definities \ref{item:enkel_fritsen_equivalent} en \ref{item:dubbelfritsen_equivalent} op pagina \pageref{item:dubbelfritsen_equivalent})}
\footnotetext[2]{\textbf{een Fritsje des nemen} is equivalent aan \textbf{Fritsen} (zie definitie \ref{item:enkel_fritsen_equivalent} op pagina \pageref{item:enkel_fritsen_equivalent})}

\newpage
\drawBar{Regels - Vervolg}
\deelhoofdstuk{Regels - Algemeen - Vervolg}
\label{sec:algemeen_einde}


\customBoxItalic{Ga zorgvuldig met de shotglaasjes, regelboekjes en kaarten om.}

\vervolgLijst{}
\item \EenSpeler mag geen shotglaasje omgooien.
\eindLijst{}

\vervolgLijst{}
\item \EenSpeler mag geen fritsgerelateerde documenten natmaken.
\eindLijst{}

\vervolgLijst{}
\item \EenSpeler mag op fritsgerelateerde documenten alleen regelgerelateerde zaken schrijven die fritsgerelateerd zijn.
\eindLijst{}

\vervolgLijst{}
\item \EenSpeler mag geen \textbf{kaarten} natmaken.
\eindLijst{}

\vervolgLijst{}
\item \EenSpeler mag geen \textbf{kaarten} op de grond laten vallen.
\eindLijst{}

\vervolgLijst{}
\item \EenSpeler mag geen schade toebrengen aan fritsgerelateerde objecten.
\eindLijst{}


\customBoxItalic{Het is belangrijk dat je serieus meespeelt. Zo niet, moet je stoppen met spelen.}

\vervolgLijst{}
\item Als \eenSpeler volgens minimaal twee derde van \alleSpelers niet serieus meedoet met het huidige potje Fritsen, worden de volgende handelingen in de gegeven volgorde uitgevoerd:
\numeriekeLijst{}
\item Één van de volgende handelingen:
\puntLijst{}
\item De \textbf{kaarten} van de desbetreffende speler worden onder de \ul{dichte} \textbf{stapel} gelegd als er een \ul{dichte} \textbf{stapel} is.
\item De \textbf{kaarten} van de desbetreffende speler worden als nieuwe \ul{dicht}e \textbf{stapel} neergelegd als er geen bestaande \ul{dichte} \textbf{stapel} is.
\eindPuntLijst{}
\item De desbetreffende speler stopt met Fritsen.
\eindNumeriekeLijst{}
\eindLijst{}

\vervolgLijst{}
\item Als \eenSpeler volgens minimaal twee derde van \alleSpelers niet serieus meedoet met het huidige potje Fritsen, mogen \alleSpelers te allen tijde de \textbf{kaarten} van de desbetreffende speler pakken en bekijken.
\eindLijst{}

\customBoxItalic{Blijf goed opletten of andere spelers geen fouten maken. Zo niet, moet je mogelijk drinken.}

\vervolgLijst{}
\item Als de \vorigeSpeler één van de regels van dit document onopgemerkt heeft overtreden en dit kenbaar maakt tijdens de beurt\footnotemark[1] van de \huidigeSpelerN, moet iedereen behalve de \vorigeSpelerN:
\puntLijst{}
\item Proosten op \quotes{supermooi Victor} en \textbf{een Victortje nemen}\footnotemark[2].
\eindPuntLijst{}
\eindLijst{}

\footnotetext[1]{zie het stroomdiagram op de achterkant van dit document}
\footnotetext[2]{\textbf{een Victortje nemen} is equivalent aan \textbf{Fritsen} (zie definitie \ref{item:enkel_fritsen_equivalent} op pagina \pageref{item:enkel_fritsen_equivalent})}

\newpage
\drawBar{Regels - Vervolg}
\deelhoofdstuk{Regels - Beginfase van het Spel}
\label{sec:beginfase_start}

\customBoxItalic{Fritsen begint nadat \Frits en \Willem zijn bepaald. Hierna schudt iemand de kaarten en geeft \Frits iedereen vijf \ul{dichte} startkaarten om vervolgens de eerste stapel open te leggen. De regels in deze sectie beschrijven het spelverloop op pagina \pageref{sec:introductie}.}

\vervolgLijst{}
\item Fritsen wordt met minimaal 2 en maximaal 10 spelers gespeeld.
\eindLijst{}

\vervolgLijst{}
\item Fritsen is begonnen zodra \Frits en \Willem zijn bepaald.
\label{regel:Willen_Frits_bepalen}
\eindLijst{}

\vervolgLijst{}
\item \Frits moet, nadat regel \ref{regel:Willen_Frits_bepalen} toegepast is, binnen 99 seconden ervoor zorgen dat \alleSpelers kunnen \textbf{vuil Fritsen}.
\eindLijst{}

\customBoxItalic{Fritsen wordt met één of twee pakken kaarten gespeeld. De kaarten moeten goed en zichtbaar worden geschud.}

\vervolgLijst{}
\item Fritsen wordt met twee \textbf{pakken kaarten}\footnotemark[1] gespeeld wanneer de groep van \alleSpelers minder dan 7 spelers bevat en meer dan de helft van \alleSpelers het spelen met twee \textbf{pakken kaarten}\footnotemark[1] goedkeuren. \label{regel:twee_pakken_minder_dan_7_personsen}
\eindLijst{}

\vervolgLijst{}
\item Fritsen wordt, tenzij regel \ref{regel:twee_pakken_minder_dan_7_personsen} toegepast is, tot en met 6 spelers met één \textbf{pak kaarten}\footnotemark[1] en tussen de 7 en 10 spelers met twee \textbf{pakken kaarten}\footnotemark[1] gespeeld.
\eindLijst{}

\vervolgLijst{}
\item Het \textbf{pak kaarten}\footnotemark[1] dat of de \textbf{pakken kaarten}\footnotemark[1] die gebruikt worden voor het huidige potje Fritsen moet(en) goed volgens en zichtbaar voor meer dan de helft van \alleSpelers geschud worden.
\label{regel:goed_zichtbaar_schudden}
\eindLijst{}

\vervolgLijst{}
\item Als \eenSpeler regel \ref{regel:goed_zichtbaar_schudden} niet naleeft, moet diegene in de gegeven volgorde:
\puntLijst{}
\item Proosten op \quotes{Frits} en \textbf{een Fritsje des nemen}\footnotemark[2].
\item Alle \textbf{kaarten} verzamelen en schudden volgens regel \ref{regel:goed_zichtbaar_schudden}.
\item Alle \textbf{kaarten} aan \Frits geven zodat diegene ze kan delen.
\eindPuntLijst{}
\eindLijst{}

\customBoxItalic{Het wordt gewaardeerd wanneer \Frits de kaarten schudt, echter kan \Frits iemand verplichten om te schudden.}

\vervolgLijst{}
\item \Frits kan \eenSpeler verplichten om de \textbf{kaarten} te schudden volgens regel \ref{regel:goed_zichtbaar_schudden} door:
\puntLijst{}
\item Dit aan te geven en \textbf{een dubbele Frits}\footnotemark[3] te nemen.
\eindPuntLijst{}
\eindLijst{}

\customBoxItalic{Onderstaande regels beschrijven stap 2 van het spelverloop op pagina \pageref{sec:introductie}.}

\vervolgLijst{}
\item \Willem moet, nadat \Frits en \Willem zijn bepaald, direct links plaatsnemen van \Frits zolang het spel niet gepauzeerd is.
\label{regel:Willen_links_van_Frits}
\eindLijst{}

\vervolgLijst{}
\item Als \Frits en \Willem regel \ref{regel:Willen_links_van_Frits} niet naleven, moeten zij:
\puntLijst{}
\item Proosten op \quotes{Frits} en \textbf{een Fritsje des nemen}\footnotemark[1].
\item Zich positioneren volgens regel \ref{regel:Willen_links_van_Frits}.
\eindPuntLijst{}
\eindLijst{}

\footnotetext[1]{zie definities \ref{item:kaarten} en \ref{item:kaarten_2} op pagina \pageref{item:kaarten_2}}
\footnotetext[2]{\textbf{een Fritsje des nemen} is equivalent aan \textbf{Fritsen} (zie definitie \ref{item:enkel_fritsen_equivalent} op pagina \pageref{item:enkel_fritsen_equivalent})}
\footnotetext[3]{\textbf{een dubbele Frits nemen} is equivalent aan twee keer \textbf{Fritsen} (zie definities \ref{item:enkel_fritsen_equivalent} en \ref{item:dubbelfritsen_equivalent} op pagina \pageref{item:dubbelfritsen_equivalent})}

\newpage
\drawBar{Regels - Vervolg}
\deelhoofdstuk{Regels - Beginfase van het spel - Vervolg}
\label{sec:beginfase_einde}

\customBoxItalic{Onderstaande regels beschrijven stap 3 en 4 van het spelverloop op pagina \pageref{sec:introductie}.}

\vervolgLijst{}
\item Alleen \Frits mag delen\footnotemark[1].
\label{regel:delen_Frits}
\eindLijst{}

\vervolgLijst{}
\item \Frits is niet verplicht om tijdens het delen\footnotemark[1] de gedeelde \textbf{kaarten} netjes, recht of gepast te positioneren.
\eindLijst{}

\vervolgLijst{}
\item Als \Frits deelt\footnotemark[1], moet diegene \alleSpelers vijf \ul{dichte} \textbf{kaarten} geven van de \ul{dichte} \textbf{stapel} door deze één voor één van bovenop de \ul{dichte} \textbf{stapel} te pakken.
\label{regel:delen_Frits_5_kaarten_1}
\eindLijst{}

\vervolgLijst{}
\item Als \Frits deelt\footnotemark[1], moet diegene \alleSpelers om de beurt één \ul{dichte} \textbf{kaart} geven.
\label{regel:delen_Frits_5_kaarten_2}
\eindLijst{}

\vervolgLijst{}
\item Nadat \Frits regels \ref{regel:delen_Frits_5_kaarten_1} en \ref{regel:delen_Frits_5_kaarten_2} heeft toegepast, moet diegene direct de \ul{bovenste} \textbf{kaart} van de \ul{dichte} \textbf{stapel} openleggen\footnotemark[2] direct naast de \ul{dichte} \textbf{stapel}.
\label{regel:eerst_kaart_open}
\eindLijst{}

\vervolgLijst{}
\item Als \Frits regel \ref{regel:eerst_kaart_open} heeft uitgevoerd, moet \Frits de \ul{dichte} \textbf{stapel} netjes, recht en gepast positioneren volgens meer dan de helft van \alleSpelersN.
\label{regel:dichte_stapel_recht_na_eerste_keer_opleggen}
\eindLijst{}


\vervolgLijst{}
\item Als \Frits een \kaart{joker}\footnotemark[3] openlegt in regel \ref{regel:eerst_kaart_open}, moet diegene nogmaals de \ul{bovenste} \textbf{kaart} van de \ul{dichte} \textbf{stapel} \ul{open} op de enige \ul{open} \textbf{stapel} leggen totdat deze geen \kaart{joker}\footnotemark[3] is.
\label{regel:eerste_kaart_open_joker}
\eindLijst{}


\vervolgLijst{}
\item \AlleSpelers moeten binnen 99 seconden \'e\'en \textbf{stapel} met \ul{dichte} \textbf{kaarten} pakken die \Frits gedeeld\footnotemark[1] heeft.
\label{regel:andere_kaarten_pakken}
\eindLijst{}

\vervolgLijst{}
\item \EenSpeler mag, tenzij diegene aan het delen is of totdat \alleSpelers een \ul{dichte} \textbf{stapel} hebben aangeraakt, maximaal \'e\'en \ul{dichte} \textbf{stapel} aanraken die is ontstaan door het toepassen van regels \ref{regel:delen_Frits_5_kaarten_1} en \ref{regel:delen_Frits_5_kaarten_2}.
\label{regel:andere_kaarten_pakken_2}
\eindLijst{}

\vervolgLijst{}
\item Als \eenSpeler regel \ref{regel:delen_Frits}, \ref{regel:delen_Frits_5_kaarten_1}, \ref{regel:delen_Frits_5_kaarten_2}, \ref{regel:eerst_kaart_open}, \ref{regel:eerste_kaart_open_joker}, \ref{regel:andere_kaarten_pakken} of \ref{regel:andere_kaarten_pakken_2} niet naleeft, moet diegene in de gegeven volgorde:
\puntLijst{}
\item Proosten op \quotes{Frits} en \textbf{een Fritsje des nemen}\footnotemark[4].
\item Alle \textbf{kaarten} verzamelen en schudden volgens regel \ref{regel:goed_zichtbaar_schudden}.
\item Alle \textbf{kaarten} aan \Frits geven zodat diegene ze opnieuw kan delen.
\eindPuntLijst{}
\eindLijst{}

\vervolgLijst{}
\item Als \eenSpeler \'e\'en of meerdere \textbf{kaarten} pakt terwijl \Frits deelt\footnotemark[1], moet diegene:
\puntLijst{}
\item Proosten op \quotes{Frits} en \textbf{een Fritsje des nemen}\footnotemark[4].
\item De gepakte \textbf{kaart(en)} houden.
\eindPuntLijst{}
\eindLijst{}



\customBoxItalic{Onderstaande regels beschrijven stap 6 van het spelverloop op pagina \pageref{sec:introductie}.}

\vervolgLijst{}
\item Tijdens het Fritsen, nadat de eerste beurt van \Willem voorbij is, zijn \alleSpelers in de richting van de klok aan de beurt\footnotemark[5].
\eindLijst{}


\footnotetext[1]{zie \texttt{stap 3} op pagina \pageref{sec:introductie}}
\footnotetext[1]{zie regels \ref{regel:kaart_netjes_1} en \ref{regel:kaart_netjes_2}}
\footnotetext[2]{zie definitie \ref{item:kaarten} op pagina \pageref{item:kaarten}}
\footnotetext[3]{\textbf{een Fritsje des nemen} is equivalent aan \textbf{Fritsen} (zie definitie \ref{item:enkel_fritsen_equivalent} op pagina \pageref{item:enkel_fritsen_equivalent})}
\footnotetext[4]{zie het stroomdiagram op de achterkant van dit document}

\newpage
\drawBar{Regels - Vervolg}
\deelhoofdstuk{Regels - Vuil Fritsen}
\label{sec:vuil_fritsen}

\customBoxItalic{\textbf{Vuil Fritsen} is het drinken en daarna inwisselen van je vijf startkaarten voordat \Willem als eerste aan de beurt\footnotemark[1] is. Een slechte hand bestaat uit meerdere kaarten met de waarde lager dan een \ul{\textbf{9}}. Je mag zo vaak \textbf{vuil Fritsen} als je wil.}

\vervolgLijst{}
\item Als \eenSpeler aan het \textbf{vuil Fritsen} is, moet diegene in de gegeven volgorde:
\puntLijst{}
\item Proosten op \quotes{vuile Frits} en \textbf{een dubbele Frits nemen}\footnotemark[2].
\item De \textbf{kaarten} van diegene \ul{dicht} op de snijplank leggen.
\item Vijf nieuwe \ul{dichte} \textbf{kaarten} van \Frits in ontvangst nemen van bovenop de \ul{dichte} \textbf{stapel} nadat \Frits de \ul{oude} \textbf{kaarten} \ul{dicht} onderop de \ul{dichte} \textbf{stapel} gelegd heeft.
\eindPuntLijst{}
\label{regel:zet_vuil_fritsen}
\eindLijst{}

\vervolgLijst{}
\item Als minimaal twee derde van \alleSpelers bepaalt dat, voordat regel \ref{regel:zet_vuil_fritsen} toegepast is en voordat \'e\'en van \alleSpelers regel \ref{regel:andere_kaarten_pakken} toegepast heeft, géén van \alleSpelers mag \textbf{vuil Fritsen}, mag géén van \alleSpelers \textbf{vuil Fritsen}.
\label{regel:skip_vuil_fritsen}
\eindLijst{}

\vervolgLijst{}
\item Als \eenSpeler ongelijk aan \Frits \textbf{kaarten} van de \ul{dichte} \textbf{stapel} pakt tijdens het toepassen van regel \ref{regel:zet_vuil_fritsen}, moet diegene:
\puntLijst{}
\item Proosten op \quotes{Frits} en \textbf{een Fritsje des nemen}\footnotemark[3].
\item De gepakte \textbf{kaart(en)} houden.
\eindPuntLijst{}
\eindLijst{}

\vervolgLijst{}
\item \EenSpeler mag alleen \textbf{vuil Fritsen} als \ul{geen} van \alleSpelers aan de beurt\footnotemark[1] geweest is en regel \ref{regel:skip_vuil_fritsen} niet toegepast is.
\eindLijst{}

\vervolgLijst{}
\item \EenSpeler mag, tenzij regel \ref{regel:skip_vuil_fritsen} toegepast is, zo vaak \textbf{vuil Fritsen} als diegene wil.
\eindLijst{}

\vervolgLijst{}
\item \EenSpeler mag alleen proosten op \quotes{vuile Frits} als diegene aan het \textbf{vuil Fritsen} is.
\eindLijst{}

\vervolgLijst{}
\item Bij \textbf{vuil Fritsen} geldt wie het eerst komt, het eerst maalt.
\eindLijst{}

\customBoxItalic{Je hebt genoeg tijd om te beslissen of je wil \textbf{vuil Fritsen}.}

\vervolgLijst{}
\item In de 99 seconden nadat regel \ref{regel:andere_kaarten_pakken} is toegepast kan alleen met de beurt van \Willem worden begonnen als \alleSpelers dit goedkeuren of regel \ref{regel:skip_vuil_fritsen} is toegepast.
\label{regel:vuile_frits_1}
\eindLijst{}

\vervolgLijst{}
\item In de 99 seconden na de laatste \textbf{vuile Frits} kan alleen met de beurt van \Willem worden begonnen als \alleSpelers dit goedkeuren.
\label{regel:vuile_frits_2}
\eindLijst{}

\vervolgLijst{}
\item Als \Willem regel \ref{regel:vuile_frits_1} of \ref{regel:vuile_frits_2} niet naleeft, moet diegene:
\puntLijst{}
\item Proosten op \quotes{Frits} en \textbf{een Fritsje des nemen}\footnotemark[3].
\item De beurt\footnotemark[1] meteen beëindigen.
\item De desbetreffende opgelegde \textbf{kaart(en)} terugnemen.
\eindPuntLijst{}
\label{regel:kaarten_terugnemen_1}
\eindLijst{}

\vervolgLijst{}
\item De beurt van \Willem is begonnen 99 seconden na de laatste \textbf{vuile Frits}.
\eindLijst{}

\footnotetext[1]{zie het stroomdiagram op de achterkant van dit document}
\footnotetext[2]{\textbf{een dubbele Frits nemen} is equivalent aan twee keer \textbf{Fritsen} (zie definities \ref{item:enkel_fritsen_equivalent} en \ref{item:dubbelfritsen_equivalent} op pagina \pageref{item:dubbelfritsen_equivalent})}
\footnotetext[3]{\textbf{een Fritsje des nemen} is equivalent aan \textbf{Fritsen} (zie definitie \ref{item:enkel_fritsen_equivalent} op pagina \pageref{item:enkel_fritsen_equivalent})}

\newpage
\drawBar{Regels - Vervolg}
\deelhoofdstuk{Regels - Beurten en Zetten}
\label{sec:beurten_en_zetten_start}

\customBoxItalic{Tijdens Fritsen is iedereen met de klok mee aan de beurt. Zolang je nog kaarten hebt, leg je elke beurt \'e\'en of twee kaarten weg en bestaat de mogelijkheid dat je twee nieuwe kaarten moet pakken. Het stroomdiagram op achterkant van dit document beschrijft hoe een beurt verloopt. Pagina \pageref{sec:regels_kort}, \pageref{zetkort:caroline} en hoofdstuk 5 in detail bevatten een overzicht van de mogelijke zetten en de bijbehorende handelingen die moeten worden uitgevoerd.}

\vervolgLijst{}
\item Als de beurt\footnotemark[1] van \eenSpeler langer dan 99 seconden duurt, moet diegene:
\puntLijst{}
\item Proosten op \quotes{Frits} en \textbf{een Fritsje des nemen}\footnotemark[2].
\item Twee \ul{dichte} \textbf{kaarten} pakken van bovenop de \ul{dichte} \textbf{stapel} wanneer dit nog niet gebeurd is in de beurt\footnotemark[1] van diegene.
\item De beurt\footnotemark[1] meteen beëindigen.
\eindPuntLijst{}
\label{regel:beurt_langer_dan_99}
\eindLijst{}

\customBoxItalic{Onderstaande regels beschrijven het stroomdiagram op de achterkant van dit document.}

\vervolgLijst{}
\item \EenSpeler mag meteen na het beginnen van een beurt\footnotemark[1] en voordat diegene geen, \'e\'en of twee \textbf{kaarten} gepakt heeft volgens regels \ref{item:geen_kaart_1}, \ref{item:geen_kaart_2} en \ref{item:geen_kaart_3} \'e\'en zet\footnotemark[4] doen.
\label{regel:zet_zonder_pakker}
\eindLijst{}

\vervolgLijst{}
\item \label{regel:twee_kaarten} Als de \huidigeSpeler geen zet\footnotemark[4] doet voordat diegene geen, \'e\'en of twee nieuw \textbf{kaarten} heeft gepakt volgens regels \ref{item:geen_kaart_1}, \ref{item:geen_kaart_2} en \ref{item:geen_kaart_3}, moet diegene in de gegeven volgorde:
\puntLijst{}
\item \textbf{Fritsen}.
\item Geen, \'e\'en of twee nieuw \textbf{kaarten} pakken van bovenop de \ul{dichte} \textbf{stapel} volgens regels \ref{item:geen_kaart_1}, \ref{item:geen_kaart_2} en \ref{item:geen_kaart_3}.
\item Één van de volgende handelingen uitvoeren:
\numeriekeLijst{}
\item Een zet\footnotemark[4] doen.
\item Een nieuwe \ul{open} \textbf{stapel} beginnen.
\eindNumeriekeLijst{}
\eindPuntLijst{}
\label{regel:fritsen_en_kaarten_pakken}
\eindLijst{}

\vervolgLijst{}
\item Proosten op \quotes{Frits} tijdens het toepassen van regel \ref{regel:twee_kaarten} is niet verplicht maar wordt wel gewaardeerd.
\eindLijst{}

\vervolgLijst{}
\item Als \eenSpeler twee \ul{dichte} \textbf{kaarten} moet pakken tijdens het toepassen van regel \ref{regel:kijken_in_dichte_stapel}, \ref{regel:beurt_langer_dan_99} of \ref{regel:twee_kaarten} wanneer er geen \ul{dichte} \textbf{kaarten} meer zijn, hoeft diegene geen \ul{dichte} \textbf{kaarten} te pakken.
\label{item:geen_kaart_1}
\eindLijst{}

\vervolgLijst{}
\item Als \eenSpeler twee \ul{dichte} \textbf{kaarten} moet pakken tijdens het toepassen van regel \ref{regel:kijken_in_dichte_stapel}, \ref{regel:beurt_langer_dan_99} of \ref{regel:twee_kaarten} wanneer er nog maar \'e\'en \ul{dichte} \textbf{kaart} is, moet diegene \'e\'en \ul{dichte} \textbf{kaart} pakken.
\label{item:geen_kaart_2}
\eindLijst{}

\vervolgLijst{}
\item Als \eenSpeler twee \ul{dichte} \textbf{kaarten} moet pakken tijdens het toepassen van regel \ref{regel:kijken_in_dichte_stapel}, \ref{regel:beurt_langer_dan_99} of \ref{regel:twee_kaarten} wanneer er meer dan \'e\'en \ul{dichte} \textbf{kaart} is, moet diegene twee \ul{dichte} \textbf{kaart} pakken.
\label{item:geen_kaart_3}
\eindLijst{}

\vervolgLijst{}
\item Als \eenSpeler méér \textbf{kaarten} dan nodig pakt van de \ul{dichte} \textbf{stapel}, moet diegene deze gepakte \textbf{kaarten} houden.
\eindLijst{}

\vervolgLijst{}
\item \EenSpeler is niet verplicht om een zet\footnotemark[4] te doen voordat en nadat diegene twee nieuwe \textbf{kaarten} heeft gepakt.
\eindLijst{}

\vervolgLijst{}
\item In de beurt\footnotemark[1] van \eenSpeler moet diegene \'e\'en zet\footnotemark[4] doen of \'e\'en nieuwe \ul{open} \textbf{stapel} beginnen als diegene nog \textbf{kaarten} heeft.
\eindLijst{}

\footnotetext[1]{zie het stroomdiagram op de achterkant van dit document}
\footnotetext[2]{\textbf{een Fritsje des nemen} is equivalent aan \textbf{Fritsen} (zie definitie \ref{item:enkel_fritsen_equivalent} op pagina \pageref{item:enkel_fritsen_equivalent})}
\footnotetext[3]{zie het stroomdiagram op de achterkant van dit document}
\footnotetext[4]{zie pagina \pageref{sec:regels_kort}, \pageref{zetkort:caroline}, \pageref{sec:zettenLang} en \pageref{sec:zettenLang_2}}

\newpage
\drawBar{Regels - Vervolg}
\deelhoofdstuk{Regels - Beurten en Zetten - Vervolg}

\customBoxItalic{Pak alleen kaart(en) van de dichte stapel als je aan de beurt bent.}

\vervolgLijst{}
\item \EenSpeler mag, tenzij diegene regel \ref{regel:zet_vuil_fritsen}, \ref{regel:beurt_langer_dan_99}, \ref{regel:fritsen_en_kaarten_pakken} of \ref{item:geen_kaart_1} aan het uitoveren is of kaarten aan het inwisselen is voor zet \ref{zet:thierry}, geen kaarten pakken van de \ul{dichte} \textbf{stapel}.
\label{regel:kaart_pakken_buiten_beurt}
\eindLijst{}

\vervolgLijst{}
\item Als \eenSpeler regel \ref{regel:kaart_pakken_buiten_beurt} niet naleeft, moet diegene:
\puntLijst{}
\item Proosten op \quotes{dubbele Frits} en \textbf{een dubbele Frits nemen}\footnotemark[1].
\item Één \ul{dichte} \textbf{kaart} pakken van bovenop de \ul{dichte} \textbf{stapel} als diegene, tijdens het niet naleven van regel \ref{regel:kaart_pakken_buiten_beurt}, één \ul{dichte} \textbf{kaart} van de \ul{dichte} \textbf{stapel} heeft gepakt.
\eindPuntLijst{}
\label{regel:kaart_pakken_buiten_beurt_straf}
\eindLijst{}

\customBoxItalic{Leg je kaarten netjes neer.}

\vervolgLijst{}
\item Als \eenSpeler een nieuwe \ul{open} \textbf{stapel} begint, moet diegene deze orthogonaal\footnotemark[2] en aansluitend positioneren ten opzichte van de bestaande \textbf{stapels}.
\label{regel:kaart_netjes_1}
\eindLijst{}

\vervolgLijst{}
\item Als \eenSpeler een \textbf{kaart} neerlegt, moet diegene deze netjes, recht en gepast positioneren volgens meer dan de helft van \alleSpelersN.
\label{regel:kaart_netjes_2}
\eindLijst{}

\vervolgLijst{}
\item Als \eenSpeler één van de opgelegde \textbf{kaarten} of één van de \textbf{kaarten} van de \ul{dichte} \textbf{stapel} verplaatst, moet diegene deze netjes, recht en gepast positioneren volgens meer dan de helft van \alleSpelersN.
\label{regel:kaart_netjes_3}
\eindLijst{}

\vervolgLijst{}
\item Als \eenSpeler één van de opgelegde \textbf{kaarten} of één van de \textbf{kaarten} van de \ul{dichte} \textbf{stapel} verplaatst, moet diegene dezelfde onderliggende verhoudingen van de \textbf{stapels} aanhouden als voor het verplaatsen.
\label{regel:kaart_netjes_4}
\eindLijst{}

\vervolgLijst{}
\item Als \eenSpeler regel \ref{regel:kaart_netjes_1}, \ref{regel:kaart_netjes_2}, \ref{regel:kaart_netjes_3} of \ref{regel:kaart_netjes_4} niet naleeft, moet diegene:
\puntLijst{}
\item Proosten op \quotes{Frits} en \textbf{een Fritsje des nemen}\footnotemark[3].
\item Alle \textbf{kaarten} op het speelbord positioneren volgens regels \ref{regel:kaart_netjes_1}, \ref{regel:kaart_netjes_2}, \ref{regel:kaart_netjes_3} en \ref{regel:kaart_netjes_4}.
\eindPuntLijst{}
\eindLijst{}

\customBoxItalic{Proost alleen op het juiste moment.}

\vervolgLijst{}
\item \EenSpeler mag alleen proosten op \quotes{kut Kim}, \quotes{Chantal}, \quotes{kut Lisa}, \quotes{Baudet}, \quotes{van der Plas}, \quotes{supermooi Erik} en \quotes{supermooi Victor} als diegene de bijbehorende zet aan het uitvoeren of regel aan het toepassen is\footnotemark[4].
\eindLijst{}

\footnotetext[1]{\textbf{een dubbele Frits nemen} is equivalent aan twee keer \textbf{Fritsen} (zie definities \ref{item:enkel_fritsen_equivalent} en \ref{item:dubbelfritsen_equivalent} op pagina \pageref{item:dubbelfritsen_equivalent})}
\footnotetext[2]{zie de speelborden op pagina \pageref{zetkort:caroline}}
\footnotetext[3]{\textbf{een Fritsje des nemen} is equivalent aan \textbf{Fritsen} (zie definitie \ref{item:enkel_fritsen_equivalent} op pagina \pageref{item:enkel_fritsen_equivalent})}
\footnotetext[4]{zie zetten \ref{zet:offer_frits}, \ref{zet:lisa} en \ref{zet:kim} op pagina \pageref{zet:kim}, zetten \ref{zet:joris} \ref{zet:thierry} en \ref{zet:caroline} op pagina \pageref{zet:joris} en regels \ref{regel:Erik_Thierry} op pagina \pageref{regel:Erik_Thierry}.}


\newpage
\drawBar{Regels - Vervolg}
\deelhoofdstuk{Regels - Beurten en Zetten - Vervolg}

\customBoxItalic{Leg alleen een kaart op wanneer dit mag.}

\vervolgLijst{}
\item \EenSpeler mag alleen een \textbf{kaart} opleggen die in lijn is met de regels en zetten.
\label{regel:ongeldige_zet}
\eindLijst{}

\vervolgLijst{}
\item Het enige moment dat \eenSpeler een \textbf{kaart} mag opleggen, tenzij elders beschreven, is tijdens de beurt\footnotemark[1] van diegene.
\label{regel:voor_beurt_opleggen}
\eindLijst{}

\vervolgLijst{}
\item \EenSpeler mag, tenzij elders beschreven, geen \textbf{kaart} onderop, tussen of bovenop de \ul{dichte} \textbf{stapel} leggen.
\label{regel:leggen_op_dichte_stapel}
\eindLijst{}

\vervolgLijst{}
\item \EenSpeler mag geen zet doen wanneer het spel is gepauzeerd\footnotemark[2].
\label{regel:tijdens_pauze_opleggen}
\eindLijst{}

\vervolgLijst{}
\item Als \eenSpeler regel \ref{regel:ongeldige_zet}, \ref{regel:voor_beurt_opleggen}, \ref{regel:leggen_op_dichte_stapel} of \ref{regel:tijdens_pauze_opleggen} niet naleeft, moet diegene:
\puntLijst{}
\item Proosten op \quotes{Frits} en \textbf{een Fritsje des nemen}\footnotemark[3].
\item De desbetreffende opgelegde \textbf{kaart(en)} terugnemen.
\eindPuntLijst{}
\label{regel:kaarten_terugnemen_2}
\eindLijst{}

\vervolgLijst{}
\item Als \eenSpeler één \textbf{kaart} of meerdere \textbf{kaarten} oplegt die niet in lijn zijn met de regels en zetten, heeft deze \textbf{kaart} of hebben deze \textbf{kaarten} geen drinkwaarde.
\eindLijst{}

\customBoxItalic{Bedenk goed waar je je kaart(en) gaat neerleggen.}

\vervolgLijst{}
\item Als \eenSpeler een \textbf{kaart} heeft neergelegd volgens regel \ref{regel:ongeldige_zet} en die \textbf{kaart} niet meer aanraakt, mag diegene die desbetreffende \textbf{kaart} niet meer op een andere \textbf{stapel} leggen.
\label{regel:kaart_in_een_keer_goed}
\eindLijst{}

\vervolgLijst{}
\item Als \eenSpeler regel \ref{regel:kaart_in_een_keer_goed} niet naleeft, moet diegene:
\puntLijst{}
\item Proosten op \quotes{Frits} en \textbf{een Fritsje te nemen}\footnotemark[3].
\item De desbetreffende \textbf{kaart} terugleggen op de originele stapel.
\eindPuntLijst{}
\eindLijst{}

\customBoxItalic{Blijf goed opletten wie er aan de beurt\footnotemark[1] is en wie geen kaarten meer heeft.}

\vervolgLijst{}
\item \EenSpeler moet van \alleSpelers weten wie aan de beurt\footnotemark[1] is en wie \textbf{uitgefritst}\footnotemark[4] zijn.
\eindLijst{}

\vervolgLijst{}
\item Als niemand van \alleSpelers weet of wil vertellen wie aan de beurt\footnotemark[1] is, is \Frits aan de beurt\footnotemark[1].
\eindLijst{}

\vervolgLijst{}
\item \EenSpeler kan \'e\'en van de \medeSpelers dwingen om naar waarheid te vertellen wie aan de beurt\footnotemark[4] is en wie \textbf{uitgefritst}\footnotemark[4] zijn door:
\puntLijst{}
\item Dit aan te geven en \textbf{een Fritsje te nemen}\footnotemark[3].
\eindPuntLijst{}
\eindLijst{}

\footnotetext[1]{zie het stroomdiagram op de achterkant van dit document}
\footnotetext[2]{zie regels \ref{regel:stilleggen_1}, \ref{regel:stilleggen_2}, \ref{regel:stilleggen_3} en \ref{regel:stilleggen_4} op pagina \pageref{regel:stilleggen_4}}
\footnotetext[3]{\textbf{een Fritsje des nemen} is equivalent aan \textbf{Fritsen} (zie definitie \ref{item:enkel_fritsen_equivalent} op pagina \pageref{item:enkel_fritsen_equivalent})}
\footnotetext[4]{\textbf{uitfritsen} is het neerleggen van je laatste \textbf{kaart(en)} (zie definitie \ref{item:uitfritsen} op pagina \pageref{item:uitfritsen})}

\newpage
\drawBar{Regels - Vervolg}
\deelhoofdstuk{Regels - Beurten en Zetten - Vervolg}
\label{sec:beurten_en_zetten_einde}

\customBoxItalic{Vergeet niet te proosten op \quotes{laatste Frits} wanneer je nog \'e\'en kaart over hebt. Verder moet je drinken als je als laatste nog kaarten hebt of als je je laatste kaart hebt neergelegd.}

\vervolgLijst{}
\item Als \eenSpeler na het opleggen nog \'e\'en \textbf{kaart} heeft, moet diegene:
\puntLijst{}
\item Binnen 9 seconden hoor- en verstaanbaar \quotes{laatste Frits} zeggen voor minimaal twee derde van \alleSpelersN.
\eindPuntLijst{}
\label{regel:laatste_frits_1}
\eindLijst{}

\vervolgLijst{}
\item Als de \huidigeSpeler na het het uitvoeren van een zet \textbf{uitgefritst}\footnotemark[2] is, moet diegene:
\puntLijst{}
\item Proosten op \quotes{Frits} en \textbf{een uittreefritsje nemen}\footnotemark[3].
\eindPuntLijst{}
\label{regel:laatste_frits_2}
\eindLijst{}

\vervolgLijst{}
\item Als \eenSpeler als laatste van \alleSpelers nog niet \textbf{uitgefritst}\footnotemark[2] is, moet diegene:
\puntLijst{}
\item Proosten op \quotes{dubbelfrits} en \textbf{een dubbele Frits nemen}\footnotemark[4].
\eindPuntLijst{}
\eindLijst{}

\customBoxItalic{Onderstaande regels beschrijven wat er gebeurt als iemands beurt\footnotemark[1] ten einde is en wanneer het spel is afgelopen.}

\vervolgLijst{}
\item Een beurt\footnotemark[1] van de \huidigeSpeler is ten einde nadat diegene regel \ref{regel:beurt_langer_dan_99}, \ref{regel:zet_zonder_pakker} of \ref{regel:twee_kaarten} heeft toegepast en al de beurtgerelateerde taken van diegene zijn verricht.
\eindLijst{}

\vervolgLijst{}
\item De beurt\footnotemark[1] van \eenSpeler is meteen ten einde wanneer diegene \textbf{uitgefritst}\footnotemark[2] is en nog geen \textbf{kaarten} heeft opgelegd.
\eindLijst{}

\vervolgLijst{}
\item Als er nog maar \textbf{[\'e\'en speler]} niet \textbf{uitgefritst}\footnotemark[2] is en niemand van \alleSpelers meer hoeft te drinken, is het spel afgelopen.
\eindLijst{}

\vervolgLijst{}
\item \alleSpelers moeten blijven meespelen totdat nog maar \textbf{[\'e\'en speler]} niet \textbf{uitgefritst}\footnotemark[2] is en niemand nog beurtgerelateerde taken hoeft uit te voeren.
\eindLijst{}

\vervolgLijst{}
\item \alleSpelers moeten blijven meespelen totdat niemand van \alleSpelers meer hoeft te drinken.
\eindLijst{}

\footnotetext[1]{zie het stroomdiagram op de achterkant van dit document}
\footnotetext[2]{\textbf{uitfritsen} is het neerleggen van je laatste \textbf{kaart(en)} (zie definitie \ref{item:uitfritsen} op pagina \pageref{item:uitfritsen})}
\footnotetext[3]{\textbf{een Fritsje des nemen} en \textbf{een uittreefritsje nemen} zijn equivalent aan \textbf{Fritsen} (zie definitie \ref{item:enkel_fritsen_equivalent} op pagina \pageref{item:enkel_fritsen_equivalent})}
\footnotetext[4]{\textbf{een dubbele Frits nemen} is equivalent aan twee keer \textbf{Fritsen} (zie definities \ref{item:enkel_fritsen_equivalent} en \ref{item:dubbelfritsen_equivalent} op pagina \pageref{item:dubbelfritsen_equivalent})}


\newpage
\drawBar{Regels - Vervolg}
\deelhoofdstuk{Regels - Jokers}
\label{sec:jokers}

\customBoxItalic{Met zetten \ref{zet:joker_1} en \ref{zet:joker_2} kan je andere spelers laten drinken door een \kaart{joker}\footnotemark[1] op te leggen. De \kaart{jokers}\footnotemark[1] worden op een aparte stapel gelegd en je mag een \kaart{joker}\footnotemark[1] niet als laatste kaart neerleggen.}

\vervolgLijst{}
\item Er moeten 2 of 3 \kaart{jokers}\footnotemark[1] per kaartspel gebruikt worden.
\eindLijst{}

\vervolgLijst{}
\item \EenSpeler mag niet \textbf{uitfritsen}\footnotemark[2] met een \kaart{joker}\footnotemark[1].
\label{regel:joker_1}
\eindLijst{}

\vervolgLijst{}
\item \EenSpeler mag niet een nieuwe \textbf{jokerstapel} beginnen als er al \'e\'en ligt.
\label{regel:joker_2}
\eindLijst{}

\vervolgLijst{}
\item \EenSpeler mag niet een \textbf{kaart} ongelijk aan een \kaart{joker}\footnotemark[1] op de \textbf{jokerstapel} leggen.
\label{regel:joker_3}
\eindLijst{}

\vervolgLijst{}
\item Als \eenSpeler regel \ref{regel:joker_1}, \ref{regel:joker_2} of \ref{regel:joker_3} niet naleeft, moet diegene:
\puntLijst{}
\item Proosten op \quotes{Frits} en \textbf{een Fritsje des nemen}\footnotemark[3].
\item De desbetreffende opgelegde \textbf{kaart(en)} terugnemen.
\eindPuntLijst{}
\label{regel:kaarten_terugnemen_3}
\eindLijst{}

\vervolgLijst{}
\item Als \Frits regel \ref{regel:eerste_kaart_open_joker} aan het toepassen is, wordt die desbetreffende \ul{open} \textbf{stapel} met \'e\'en of meer \kaart{jokers} niet als \textbf{jokerstapel} beschouwd.
\eindLijst{}

\customBoxItalic{De \textbf{jokerstapel} bevindt zich direct naast de dichte stapel.}

\vervolgLijst{}
\item Als \eenSpeler een \kaart{joker}\footnotemark[1] neerlegt, moet deze zich direct naast de \ul{dichte} \textbf{stapel} bevinden.
\label{regel:plaats_joker}
\eindLijst{}

\vervolgLijst{}
\item \EenSpeler mag niet een \textbf{kaart} opleggen waardoor de \textbf{jokerstapel} zich niet direct naast de \ul{dichte} \textbf{stapel} kan bevinden.
\label{regel:joker_4}
\eindLijst{}

\vervolgLijst{}
\item Als \eenSpeler regel \ref{regel:plaats_joker} of \ref{regel:joker_4} niet naleeft, moet diegene:
\puntLijst{}
\item Proosten op \quotes{Frits} en \textbf{een Fritsje des nemen}\footnotemark[3].
\item De desbetreffende opgelegde \textbf{kaart(en)} terugnemen.
\eindPuntLijst{}
\label{regel:kaarten_terugnemen_4}
\eindLijst{}

\customBoxItalic{Nadat \Frits iedereen 5 dichte startkaarten gegeven heeft, legt \Frits de eerste stapel open. In het geval van een joker, legt \Frits een nieuwe kaart eroverheen en herhaalt dit tot deze nieuwe kaart geen joker meer is. In deze situatie hoeft niemand te drinken.}

\vervolgLijst{}
\item Als \Frits regel \ref{regel:eerste_kaart_open_joker} aan het toepassen is, hebben de opgelegde \kaart{joker(s)}\footnotemark[1] geen drinkwaarde en worden ze niet gezien als een \textbf{jokerstapel}.
\eindLijst{}

\footnotetext[1]{zie definities \ref{item:kaarten} en \ref{item:kaarten_2} op pagina \pageref{item:kaarten_2}}
\footnotetext[2]{\textbf{uitfritsen} is het neerleggen van je laatste \textbf{kaart(en)} (zie definitie \ref{item:uitfritsen} op pagina \pageref{item:uitfritsen})}
\footnotetext[3]{\textbf{een Fritsje des nemen} is equivalent aan \textbf{Fritsen} (zie definitie \ref{item:enkel_fritsen_equivalent} op pagina \pageref{item:enkel_fritsen_equivalent})}

\newpage
\drawBar{Regels - Vervolg}
\deelhoofdstuk{\proLabel Regels - Thierry Baudet en Jesse Klaver}
\label{sec:thierry}

\customBox{\textit{Zet }\ref{zet:thierry}\footnotemark[1] (Thierry)\textit{ is het omruilen van je kaarten midden in het potje Fritsen nadat je een \kaart{6} op een \kaart{vrouw} hebt gelegd, hebt geproost op \quotes{Baudet} en dubbel hebt gedronken. Thierry Baudet wil het partijkartel omverwerpen. Je weggelegde kaarten zijn het partijkartel en de nieuwe gevestigde orde is de nieuwe set kaarten die je krijgt van \FritsN.}}

\vervolgLijst{}
\item \EenSpeler mag niet \textbf{uitfritsen}\footnotemark[2] met zet \ref{zet:thierry}\footnotemark[1].
\label{regel:thierry_1}
\eindLijst{}

\vervolgLijst{}
\item Als \eenSpeler regel \ref{regel:thierry_1} niet naleeft, moet diegene:
\puntLijst{}
\item Proosten op \quotes{Frits} en \textbf{een Fritsje des nemen}\footnotemark[3].
\item De desbetreffende opgelegde \textbf{kaart} terugnemen.
\eindPuntLijst{}
\label{regel:kaarten_terugnemen_5}
\eindLijst{}

\customBox{\textit{Jesse Klaver wil niet dat Thierry Baudet de macht krijgt. Wanneer je zet }\ref{zet:thierry}\footnotemark[1]\textit{ aan het uitvoeren bent, kan \eenSpeler het inruilen van jouw hand voorkomen door direct na het opleggen van jouw \kaart{6} een \kaart{klaveren 3} eroverheen te leggen.}}

\vervolgLijst{}
\item \label{zet:Jesse} Tussen het moment dat de \huidigeSpeler zet \ref{zet:thierry}\footnotemark[1] heeft gestart en nieuwe \textbf{kaarten} heeft gekregen van \FritsN, mag \eenSpeler een \kaart{Jesse Klaver}\footnotemark[4] direct op de net opgelegde \kaart{Thierry}\footnotemark[5] leggen.
\eindLijst{}

\vervolgLijst{}
\item \label{zet:Jesse_2} Als \eenSpeler regel \ref{zet:Jesse} heeft toegepast, moet de \huidigeSpelerN:
\puntLijst{}
\item Alle weggelegde \textbf{kaarten} die bedoeld zijn voor het inwisselen terugnemen.
\item Stoppen met het inwisselen van de desbetreffende kaarten.
\eindPuntLijst{}
\eindLijst{}

\vervolgLijst{}
\item \Frits mag geen \textbf{kaarten} bekijken als diegene die desbetreffende \textbf{kaarten} voor een \andereSpeler aan het inwisselen is.
\eindLijst{}

\vervolgLijst{}
\item \label{regel:Erik_Thierry} Als \eenSpeler in dezelfde beurt\footnotemark[6] volgens regel \ref{zet:Jesse} een \kaart{Jesse Klaver}\footnotemark[4] op een \kaart{Thierry}\footnotemark[5] legt, moet diegene:
\puntLijst{}
\item Proosten op \quotes{supermooi Erik} en \textbf{een Erikje nemen}\footnotemark[7].
\eindPuntLijst{}
\eindLijst{}

\vervolgLijst{}
\item \label{regel:Erik_Thierry_2} \EenSpeler moet, tenzij diegene regel \ref{regel:Erik_Thierry} aan het toepassen is, te allen tijde proosten op \quotes{Baudet} en \textbf{een Thierry'tje nemen}\footnotemark[6] bij het toepassen van zet \ref{zet:thierry}\footnotemark[1].
\eindLijst{}

\customBoxItalic{Je hebt genoeg tijd om de \kaart{klaveren 3} over de net opgelegde \kaart{6} heen te leggen.}

\vervolgLijst{}
\item \Frits mag pas 9 seconden nadat \eenSpeler zet \ref{zet:thierry}\footnotemark[1] heeft gestart \textbf{kaarten} gaan inwisselen.
\eindLijst{}

\footnotetext[1]{zie pagina \pageref{zetkort:thierry} en \pageref{zet:thierry}}
\footnotetext[2]{\textbf{uitfritsen} is het neerleggen van je laatste \textbf{kaart(en)} (zie definitie \ref{item:uitfritsen} op pagina \pageref{item:uitfritsen})}
\footnotetext[3]{\textbf{een Fritsje des nemen} is equivalent aan \textbf{Fritsen} (zie definitie \ref{item:enkel_fritsen_equivalent} op pagina \pageref{item:enkel_fritsen_equivalent})}
\footnotetext[4]{de \textbf{kaart} \kaart{Jesse Klaver} is equivalent aan een \kaart{klaveren 3} (zie pagina \pageref{sec:kaartnamen})}
\footnotetext[5]{de \textbf{kaart} \kaart{Thierry} is equivalent aan elke \kaart{6} (zie pagina \pageref{sec:kaartnamen})}
\footnotetext[6]{zie het stroomdiagram op de achterkant van dit document}
\footnotetext[7]{\textbf{een Erikje nemen} en \textbf{een Thierry'tje nemen} zijn equivalent aan \textbf{Dubbelfritsen} (zie definitie \ref{item:dubbelfritsen_equivalent} op pagina \pageref{item:dubbelfritsen_equivalent})}

\newpage
\drawBar{Regels - Vervolg}
\deelhoofdstuk{\proLabel Regels - Caroline van der Plas en Jesse Klaver}
\label{sec:caroline}

\customBox{\textit{Zet \ref{zet:caroline}\footnotemark[1]} is het blokkeren van open stapels voor \andereSpelers door het neerzetten van een shotglaasje op \'e\'en van open stapels nadat je een \kaart{6} op een \kaart{boer} hebt gelegd. Zie de speelborden op pagina \pageref{zetkort:caroline}. Caroline van der Plas is aan het demonstreren tegen het stikstofbeleid. De geblokkeerde stapels zijn de versperde wegen: je kan hier geen kaarten opleggen.}

\vervolgLijst{}
\item \EenSpeler mag niet \textbf{uitfritsen}\footnotemark[2] met zet \ref{zet:caroline}\footnotemark[1].
\label{regel:caroline_uitfritsen}
\eindLijst{}

\vervolgLijst{}
\item \EenSpeler mag niet een \textbf{kaart} opleggen op een \ul{open} \textbf{stapel} met daarop een shotglaasje of die direct orthogonaal\footnotemark[4] verbonden is met een \ul{open} \textbf{stapel} met daarop een shotglaasje.
\label{regel:orthogonaal_verkeerd_opleggen}
\eindLijst{}

\vervolgLijst{}
\item Als \eenSpeler regel \ref{regel:caroline_uitfritsen} of \ref{regel:orthogonaal_verkeerd_opleggen} niet naleeft, moet diegene:
\puntLijst{}
\item Proosten op \quotes{Frits} en \textbf{een Fritsje des nemen}\footnotemark[3].
\item De desbetreffende opgelegde \textbf{kaart} terugnemen.
\eindPuntLijst{}
\label{regel:kaarten_terugnemen_7}
\eindLijst{}

\customBox{\textit{Vergeet niet om het shotglaasje bij je volgende beurt\footnotemark[5] van het speelbord te halen.}}

\vervolgLijst{}
\item \label{regel:caroline_weghalen} Als in de vorige beurt\footnotemark[4] van de \huidigeSpeler zet \ref{zet:caroline}\footnotemark[1] uitgevoerd is en diegene een shotglaasje op \'e\'en van de \ul{dichte} \textbf{stapels} heeft gezet, moet in de huidige beurt\footnotemark[4] het desbetreffende shotglaasje als eerste van de desbetreffende \textbf{stapel} gehaald worden.
\eindLijst{}

\vervolgLijst{}
\item Als \eenSpeler regel \ref{regel:caroline_weghalen} niet naleeft, moet diegene:
\puntLijst{}
\item Proosten op \quotes{Frits} en \textbf{een Fritsje des nemen}\footnotemark[3].
\item Het desbetreffende shotglaasje van de desbetreffende \textbf{kaart} halen.
\eindPuntLijst{}
\eindLijst{}

\vervolgLijst{}
\item Als niemand van \alleSpelers weet of wil vertellen wie het shotglaasje geplaatst heeft tijdens het uitvoeren van zet \ref{zet:caroline}\footnotemark[1], moeten de volgende handelingen uitgevoerd worden:
\puntLijst{}
\item \AlleSpelers proosten op \quotes{Frits} en \textbf{nemen een Fritsje des}\footnotemark[3].
\item Het desbetreffende shotglaasje wordt van de desbetreffende \textbf{kaart} gehaald.
\eindPuntLijst{}
\eindLijst{}

% \customBox{\textit{De politie wil niet dat Caroline van der Plas ervoor zorgt dat alle wegen ontoegankelijk zijn. Wanneer je zet \ref{zet:caroline}\footnotemark[1] aan het uitvoeren ben, kan \eenSpeler het blokkeren van open \textbf{stapels} voorkomen door het direct na het opleggen van jouw \kaart{boer} een \kaart{ruiten 4} eroverheen te leggen.}}

% \vervolgLijst{}
%     \item \label{zet:Politie} Tussen het moment dat de \huidigeSpeler zet \ref{zet:caroline}\footnotemark[1] heeft gestart en een shotglaasje op \'e\'en van de dicht \textbf{stapels} heeft geplaatst, mag \eenSpeler een \kaart{ruiten 4} direct op de net opgelegde \kaart{Caroline}\footnotemark[3] leggen.
%  \eindLijst{}

%  \vervolgLijst{}   
%     \item \label{zet:Politie_2} Als \eenSpeler regel \ref{zet:Politie} heeft toegepast, moet de \huidigeSpeler in de gegeven volgorde:
%      \numeriekeLijst{}
%         \item{\'E\'en van deze handelingen}
%         \puntLijst{}
%             \item Geen dichte \textbf{kaarten} pakken als er geen dichte \textbf{kaarten} meer zijn.
%             \item \'E\'en \ul{dichte} \textbf{kaart} pakken als er \'e\'en \ul{dichte} \textbf{kaart} is.
%             \item Twee \ul{dichte} \textbf{kaarten} pakken als er meer dan \'e\'en \ul{dichte} \textbf{kaart} is. 
%         \eindPuntLijst{}
%         \item Stoppen met het plaatsen van het shotglaasje op \'e\'en van de \ul{dichte} \textbf{stapels}. 
%     \eindNumeriekeLijst{}
% \eindLijst{}

% \vervolgLijst{}
%     \item De \huidigeSpeler mag pas 9 seconden nadat diegene zet \ref{zet:caroline}\footnotemark[1] heeft gestart een shotglaasje plaatsen op \'e\'en van de \ul{dichte} \textbf{stapels}. 
% \eindLijst{}

% \vervolgLijst{}
%     \item \label{zet:Erik_Caroline} Als \eenSpeler in dezelfde beurt volgens regel \ref{zet:Politie} een \kaart{ruiten 4} op een \kaart{Caroline}\footnotemark[3] legt, moet diegene:
%     \puntLijst{}
%         \item Proosten op \quotes{supermooi Erik} en \textbf{een Erikje nemen}\footnotemark[4].
%     \eindPuntLijst{}
% \eindLijst{}

% \vervolgLijst{}
%     \item \label{zet:Erik_Caroline_2} \EenSpeler moet, tenzij diegene regel \ref{zet:Erik_Caroline} aan het toepassen is, te allen tijde proosten op \quotes{van der Plas} en \textbf{een Caroline'tje nemen}\footnotemark[4] bij het toepassen van zet \ref{zet:caroline}\footnotemark[1].
% \eindLijst{}

\footnotetext[1]{zie pagina \pageref{zetkort:caroline} en \pageref{zet:caroline}}
\footnotetext[2]{\textbf{uitfritsen} is het neerleggen van je laatste \textbf{kaart(en)} (zie definitie \ref{item:uitfritsen} op pagina \pageref{item:uitfritsen})}
\footnotetext[3]{\textbf{een Fritsje des nemen} is equivalent aan \textbf{Fritsen} (zie definitie \ref{item:enkel_fritsen_equivalent} op pagina \pageref{item:enkel_fritsen_equivalent})}
\footnotetext[4]{zie de speelborden op pagina \pageref{zetkort:caroline}}
\footnotetext[5]{zie het stroomdiagram op de achterkant van dit document}

% \footnotetext[3]{de \textbf{kaart} \kaart{Caroline} is equivalent aan elke \kaart{boer} (zie pagina \pageref{sec:kaartnamen})}
% \footnotetext[4]{\textbf{een Erikje nemen} en \textbf{een Caroline'tje nemen} zijn equivalent aan \textbf{Dubbelfritsen} (zie definitie \ref{item:dubbelfritsen_equivalent} op pagina \pageref{item:dubbelfritsen_equivalent})}

\newpage
\drawBar{Regels - Vervolg}
\deelhoofdstuk{\proLabel Regels - Caroline van der Plas en Jesse Klaver - Vervolg}

\customBox{\textit{Er wordt maar \'e\'en shotglaasje gebruikt om stapels te blokkeren.}}

\vervolgLijst{}
\item \label{regel:caroline_een_shotglaasje} Er bevindt zich maximaal \'e\'en shotglaasje op \'e\'en van alle \textbf{stapels}
\label{regel:shotglaasje_te_veel}
\eindLijst{}

\vervolgLijst{}
\item \label{regel:caroline_onjuist_plaatsen_1} \EenSpeler mag geen shotglaasje plaatsen op de \ul{dichte} \textbf{stapel}.
\label{regel:shotglaasje_op_dichte_stapel}
\eindLijst{}


\vervolgLijst{}
\item Als \eenSpeler regel \ref{regel:shotglaasje_te_veel} of \ref{regel:shotglaasje_op_dichte_stapel} niet naleeft, moet diegene:
\puntLijst{}
\item Proosten op \quotes{Frits} en \textbf{een Fritsje des nemen}\footnotemark[1].
\item Het desbetreffende shotglaasje van de desbetreffende \textbf{stapel} halen.
\eindPuntLijst{}
\eindLijst{}


\customBox{\textit{Raak en verplaats niet het shotglaasjes van een andere speler}}

\vervolgLijst{}
\item \label{regel:shotglaasje_aanraken_1} Als er zich een shotglaasje op \'e\'en van de \textbf{stapels} bevindt, mag deze eenmalig worden verplaats door de \huidigeSpeler tijdens het uitvoeren van zet \ref{zet:caroline}\footnotemark[2].
\eindLijst{}

\vervolgLijst{}
\item \label{regel:shotglaasje_aanraken_2} Als er zich een shotglaasje op \'e\'en van de \textbf{stapels} bevindt, mag deze eenmalig worden verplaats door de \huidigeSpeler als in de vorige beurt van diegene zet \ref{zet:caroline}\footnotemark[2] uitgevoerd is en daarna niemand van \alleSpelers zet \ref{zet:caroline}\footnotemark[2] heeft uitgevoerd.
\eindLijst{}

\vervolgLijst{}
\item Als \eenSpeler regel \ref{regel:shotglaasje_aanraken_1} of \ref{regel:shotglaasje_aanraken_2} niet naleeft, moet diegene:
\puntLijst{}
\item Proosten op \quotes{Frits} en \textbf{een Fritsje des nemen}\footnotemark[1].
\item Het desbetreffende shotglaasje terugplaatsen naar de vorige positie.
\eindPuntLijst{}
\eindLijst{}

\customBox{\textit{Plaats alleen een shotglaasje in je eigen beurt\footnotemark[3] om stapels te blokkeren.}}

\vervolgLijst{}
\item Als \eenSpeler een shotglaasje op een \textbf{stapel} plaatst als diegene niet zet \ref{zet:caroline}\footnotemark[2] aan het uitvoeren is, moet diegene:
\puntLijst{}
\item Proosten op \quotes{Frits} en \textbf{een Fritsje des nemen}\footnotemark[1].
\item Één van de volgende handelingen uitvoeren:
\numeriekeLijst{}
\item Het desbetreffende shotglaasje van de desbetreffende \textbf{stapel} halen als er zich nog geen shotglaasje op \'e\'en van de \ul{open} \textbf{stapels} bevindt.
\item Het desbetreffende shotglaasje terugzetten op de vorige \textbf{stapel} als er zich al een shotglaasje op \'e\'en van de \ul{open} \textbf{stapels} bevindt.
\eindNumeriekeLijst{}
\eindPuntLijst{}
\eindLijst{}

\footnotetext[1]{\textbf{een Fritsje des nemen} is equivalent aan \textbf{Fritsen} (zie definitie \ref{item:enkel_fritsen_equivalent} op pagina \pageref{item:enkel_fritsen_equivalent})}
\footnotetext[2]{zie pagina \pageref{zetkort:caroline} en \pageref{zet:caroline}}
\footnotetext[3]{zie het stroomdiagram op de achterkant van dit document}
