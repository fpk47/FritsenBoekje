\drawBarFirstPage{Introductie}
\section*{Introductie}
\drawParagraph{Het spel in het kort}
$-$ Fritsen is een drankspel met kaarten dat gespeeld wordt met minimaal 2 en maximaal 10 personen. \newline $-$ Om de beurt leg je een kaart neer en moeten jullie mogelijk drinken. \newline $-$ Als je \ul{\texttt{ADTEN}} zegt moet je proosten op \quotes{Frits} en een shotglaasje bier drinken. 
\vspace*{-0.22cm} 

\drawParagraph{Doel van het spel} 
$-$ Je kaarten \ul{zo snel mogelijk} wegleggen en \ul{zo min mogelijk} drinken.  

\vspace*{-0.22cm} 

\drawParagraph{Duur van het spel}
$-$ Tussen de 15 en 30 minuten.

\vspace*{-0.22cm} 

\drawParagraph{Benodigdheden}
$-$ Voldoende bier en shotglaasjes. \newline $-$ Tot en met 6 personen \ul{\'e\'en} pak kaarten en voor meer dan 6 personen \ul{twee} pakken kaarten.

\vspace*{-0.22cm} 

\drawParagraph{Soorten stapels kaarten}
$-$ \'E\'en dichte stapel kaarten, een open jokerstapel en meerdere open stapels voor niet-jokers.

\noindent\rule{\textwidth}{1pt}
\centerline{\textit{"\textbf{Fritsen} of \textbf{een Fritsje nemen} is het drinken van één shotglaasje bier."}} \\
\centerline{\textit{"\textbf{Dubbel Fritsen} of \textbf{een dubbele Frits nemen} is het drinken van twee shotglaasjes bier."}} \vspace*{-0.7cm}  \\
\noindent\rule{\textwidth}{1pt}

%\vspace*{0.2cm}
%\centerline{\large{\textbf{Met dit hoofdstuk en het stroomdiagram op de achterkant kun je Fritsen}}}
\vspace*{-0.45cm}

\section*{Spelverloop}
\label{sec:introductie}
\begin{minipage}[t]{.09\textwidth}
\texttt{Stap 1}:
\end{minipage}
\hfill
\begin{minipage}[t]{.91\textwidth}
Bepaal samen, met behulp van de foto's op de achterkant van dit document, wie het meest op Frits Kastelein (dit is \textbf{[Frits]}) en Willem Vaandrager (dit is \textbf{[Willem}]) lijkt. \\
\end{minipage}

%\vspace*{-0.13cm}
\noindent
\begin{minipage}[t]{.09\textwidth}
\texttt{Stap 2}:
\end{minipage}
\hfill
\begin{minipage}[t]{.91\textwidth}
\Willem neemt direct links van \textbf{[Frits]} plaats. 
\end{minipage}
\\

%\vspace*{-0.13cm}
\noindent
\begin{minipage}[t]{.09\textwidth}
\texttt{Stap 3}:
\end{minipage}
\hfill
\begin{minipage}[t]{.91\textwidth}
Iemand schudt de kaarten en \Frits geeft iedereen vijf \ul{dichte} startkaarten die je vervolgens in je hand mag nemen en bekijken. \Frits legt de overige kaarten dicht op tafel. \\
\end{minipage}

%\vspace*{-0.13cm}
\noindent
\begin{minipage}[t]{.09\textwidth}
\texttt{Stap 4}:
\end{minipage}
\hfill
\begin{minipage}[t]{.91\textwidth}
\Frits begint een nieuwe open stapel met de bovenste kaart van de \ul{dichte} stapel. Als die kaart een \textbf{\ul{joker}} is, herhaalt \Frits \texttt{stap 4} en voert \'e\'en van de volgende acties uit:
\puntLijst{}
    \item begin een nieuwe \ul{open} jokerstapel als er nog geen jokerstapel ligt.
    \item leg de \kaart{joker} op de bestaande jokerstapel.
\eindPuntLijst{}
\end{minipage}

\vspace*{+0.35cm} 

\noindent
\begin{minipage}[t]{.09\textwidth}
\texttt{Stap 5}:
\end{minipage}
\hfill
\begin{minipage}[t]{.91\textwidth}
Voordat het spel daadwerkelijk begint, kun je nog van kaarten wisselen. Een slechte hand bestaat uit meerdere kaarten met een waarde lager dan een \kaart{9}. Je kunt zo vaak als je wilt je kaarten inwisselen (dit heet \textbf{vuil Fritsen}) door:
\numeriekeLijst{}
    \item Te proosten op "\textit{vuile Frits}" en \textbf{een dubbele Frits te nemen}.
    \item Je startkaarten \ul{dicht} op de tafel te leggen.
    \item \Frits legt de kaarten onder de \ul{dichte} stapel en geeft je vijf nieuwe startkaarten.
\eindNumeriekeLijst{}
\end{minipage}
\vspace*{+0.3cm} 
    
%\vspace*{-0.10cm}
\noindent
\begin{minipage}[t]{.09\textwidth}
\texttt{Stap 6}:
\end{minipage}
\hfill
\begin{minipage}[t]{.91\textwidth}
\Willem is als eerste aan de beurt. Hierna zijn de andere spelers met de klok mee aan de beurt. \textbf{\textit{Het stroomdiagram op de achterkant van dit document beschrijft hoe een beurt verloopt}}. \\
\end{minipage}

%\vspace*{-0.13cm}
\noindent
\begin{minipage}[t]{.09\textwidth}
\texttt{Stap 7}:
\end{minipage}
\hfill
\begin{minipage}[t]{.91\textwidth}
Als je tijdens het spelen \ul{als laatste} nog kaarten hebt, moet je proosten op \quotes{dubbel Frits} en \textbf{een dubbele Frits nemen}. Daarna is het spel afgelopen.  \\
\end{minipage}

%\vspace*{-0.15cm}
%\noindent
%\begin{minipage}[t]{.09\textwidth}
%\texttt{ Extra}:
%\end{minipage}
%\hfill
%\begin{minipage}[t]{.91\textwidth}
%Tijdens het Fritsen kun je zeggen dat je \quotes{uitfritsgarantie} hebt. Dit betekent dat je zonder extra kaarten te pakken kan uitkomen. Bluffen is toegestaan.  
%\end{minipage}

\newpage
\drawBar{Introductie - Vervolg}
\vspace*{0.2cm}
\deelhoofdstuk{Zetten - Andere spelers drinken \ul{niet}}
\noindent
\begin{minipage}[t]{.48\textwidth}
  \zetKortKort{0.40}{1}{Een kaart op een \ul{lagere} kaart van \ul{hetzelfde} symbool leggen.}
  \zetKort{0.40}{2}{Een kaart op een kaart \\\ul{\'e\'en} hoger van \ul{hetzelfde} symbool leggen}{Proost op \quotes{offer Frits} en \textbf{neem een Fritsje}.}
\end{minipage}% This must go next to `\end{minipage}`
\hfill \vrule \hfill
\begin{minipage}[t]{.48\textwidth}
  \zetKortKort{0.40}{3}{Een \kaart{2} op een \kaart{aas} van \newline \ul{hetzelfde} symbool leggen.}
  \zetKortKort{0.40}{4}{Een \kaart{9} op een kaart ongelijk aan een \kaart{joker} leggen.}
\end{minipage}

\deelhoofdstuk{Zet - Kaarten inwisselen}
\noindent

\zetLang{0.20}{5}{Een \kaart{6} op een \kaart{vrouw} leggen}{Leg je kaarten \ul{dicht} op tafel, proost op \quotes{Baudet} en \textbf{neem een dubbele Frits}. \Frits legt de kaarten onderop de dichte stapel en geeft je evenveel nieuwe kaarten.}{\ul{\texttt{LET OP}}: Niet uitgaan met deze zet!}

\deelhoofdstuk{Zetten - Andere spelers \ul{moeten} drinken}
\label{sec:regels_kort}

\noindent
\begin{minipage}[t]{.48\textwidth}
 \zetKort{0.40}{7}{Een \kaart{koning} op een \kaart{harten aas} leggen of andersom}{Andere spelers proosten op \quotes{kut Lisa} en \\\textbf{Fritsen}.}
\end{minipage}
\hfill \vrule \hfill
\begin{minipage}[t]{.48\textwidth}
\zetKort{0.40}{6}{Een \kaart{rode vrouw} op een \kaart{klaveren boer} leggen of andersom}{Andere spelers proosten op \quotes{Chantal} en \\\textbf{Fritsen}.}
\end{minipage}

\vspace{+0.5cm}

\zetLang{0.20}{9}{Een \kaart{vrouw} op een \kaart{vrouw} leggen}{Andere spelers proosten op \quotes{kut Kim} en \textbf{Fritsen}.}{\ul{\texttt{LET OP}}: Als de \kaart{vrouw} op een stapel met drie achtereenvolgende \kaart{vrouwen} wordt gelegd, moeten de andere spelers \textbf{een dubbele Frits nemen}.}

\zetLang{0.20}{8}{Een \kaart{9} op een \kaart{9} leggen}{Andere spelers proosten op \quotes{iedereen dubbel Frits} en \textbf{nemen een dubbele Frits}.}{}

\zetLang{0.20}{10}{Een \kaart{joker} op de jokerstapel leggen}{Begin een nieuwe jokerstapel als er nog geen jokerstapel ligt.}{\ul{\texttt{LET OP}}: Er is maar \ul{\'e\'en} jokerstapel!\\ \ul{\texttt{LET OP}}: Niet uitgaan met deze zet!}

\vspace{+0.4cm}

\centerline{\Large{\textbf{De rest van dit document bevat de volledige spelregels}}}


