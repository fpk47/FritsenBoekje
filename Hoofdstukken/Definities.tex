\newpage
\drawBar{}

\section*{1. Definities}

\beginABCLijst{1}
\item \label{itm:1A} Uitspraken, acties en belangrijke begrippen worden \textbf{vetgedrukt} weergegeven.

\item Kaarten worden \ul{\textbf{vetgedrukt en onderstreept}} weergegeven.

\item Personen worden \textbf{[vetgedrukt en tussen vierkante haken]} weergegeven.

\item Een \textbf{stapel} is één of meer op elkaar gelegde kaarten.

\item De vier typen kaartspelsymbolen of \textbf{symbolen} zijn \ul{\textbf{harten}}, \ul{\textbf{klaveren}}, \ul{\textbf{ruiten}} en \ul{\textbf{schoppen}}. 

\item \label{def:fritsen} \textbf{Fritsen} of \textbf{een Fritsje nemen} is het drinken van minimaal 3 cl bier van minimaal 4 procent alcohol uit een shotglaasje of het drinken van een equivalente hoeveelheid alcohol uit een ander glas.

\item \label{item:enkel_fritsen_equivalent}\textbf{Een Fritsje des nemen}, \textbf{een uittreefritsje nemen}, \textbf{een Lisaatje nemen}, \textbf{een Kimmetje nemen}, \textbf{een Jorisje nemen} zijn equivalent aan \textbf{Fritsen} en \textbf{een Fritsje nemen}. 

\item \label{item:dubbelfritsen_equivalent} \textbf{Dubbelfritsen} en \textbf{een dubbele Frits nemen} zijn equivalent aan twee keer \textbf{Fritsen}.

\item \textbf{Een dubbele Kim nemen}, \textbf{een Erikje nemen} en \textbf{een Thierry'tje nemen} zijn equivalent aan \textbf{dubbelfritsen} en \textbf{een dubbele Frits nemen}.

\item \textbf{[Alle spelers]} zijn diegenen die momenteel aan het Fritsen zijn.

\item \textbf{[Een speler]} is één van \textbf{[alle spelers]}. 

\item De \huidigeSpeler is diegene die momenteel aan de beurt\footnotemark[1] is.  

\item Een \andereSpeler is iemand die momenteel \ul{niet} aan de beurt\footnotemark[1] is. 

\item De \andereSpelers zijn diegenen die momenteel \ul{niet} aan de beurt\footnotemark[1] zijn. 

\item De \volgendeSpeler van \eenSpeler is diegene die na hem/haar aan de beurt\footnotemark[1] is. 

\item De \medeSpelers van \eenSpeler zijn \alleSpelers behalve hem/haar.

\item \textbf{[Frits]} is diegene die het meest op \textit{Frits Kastelein}\footnotemark[2] lijkt volgens meer dan de helft van \textbf{[alle spelers]\footnotemark[3]}.

\item \textbf{[Willem]} is diegene die het meest op \textit{Willem Vaandrager}\footnotemark[2] lijkt volgens meer dan de helft van \textbf{[alle spelers]\footnotemark[3]}.

\item \label{item:kim}\textbf{[Kim]} is \textit{Kim de Boer}.

\item \label{item:lisa}\textbf{[Lisa]} is \textit{Lisa de Haan}.

\item \label{item:uitfritsen} \textbf{Uitfritsen} is het neerleggen van de laatste kaart(en) van \textbf{[een speler]}.

\item  \label{item:uitfritsgarantie} \textbf{Uitfritsgarantie} hebben is het niet meer nieuwe kaarten hoeven te pakken en elke beurt\footnotemark[1] kaart(en) kunnen wegleggen.

\item \label{item:uitklaver} \textbf{Uitklaveren} is het \textbf{uitfritsen} met een \kaart{klaveren 3}\footnotemark[4].

\item \label{item:kaarten} \'E\'en \textbf{pak kaarten} bestaat uit: 
    \puntLijst{}
        \item de unieke set kaarten met \textbf{symbolen} schoppen, harten klaveren en ruiten met namen \kaart{2}, \kaart{3}, \kaart{4}, \kaart{5}, \kaart{6}, \kaart{7}, \kaart{8}, \kaart{9}, \kaart{10}, \kaart{boer}, \kaart{vrouw}, \kaart{koning} en \kaart{aas}.
        \item twee jokers en, indien aanwezig, een bridge-kaart.
    \eindPuntLijst{}

\item \label{item:kaarten_2} De waarde van de kaarten in oplopende volgorde zijn \kaart{2}, \kaart{3}, \kaart{4}, \kaart{5}, \kaart{6}, \kaart{7}, \kaart{8}, \kaart{9}, \kaart{10}, \kaart{boer}, \kaart{vrouw}, \kaart{koning} en \kaart{aas}. De rest van de kaarten zijn \kaart{jokers}.

\item \textbf{Dichtfritsen} is een \textbf{stapel} compleet dichtleggen, d.w.z., op die \textbf{stapel} kan alleen nog maar een \ul{\textbf{9}} of een \ul{\textbf{harten aas}} worden gelegd.



\item \label{item:kanonskogel} Een \textbf{kanonskogel} is een shotglaasje van minimaal 3 cl met 50 procent Sambuca en 50 procent Goldstrike met, mits aanwezig, drie druppels tabasco.

\eindABCLijst

\footnotetext[1]{zie het stroomdiagram op de achterkant van dit document}
\footnotetext[2]{zie de foto's op de achterkant van dit document}
\footnotetext[3]{gebruik bij twijfel regel \ref{item:beslissingCriteria}}
\footnotetext[4]{zie regels \ref{zet:Jesse}, \ref{zet:Jesse_2} en \ref{zet:Erik}}