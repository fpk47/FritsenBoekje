\newpage
\drawBar{}

\vspace{-0.5cm}
\section*{1. Definities}

\beginABCLijst{1}
\item Een handeling of belangrijk begrip wordt \textbf{vetgedrukt} weergegeven.

\item Een kaart wordt \kaart{vetgedrukt en onderstreept} weergegeven.

\item Één of meerdere spelers worden \textbf{[vetgedrukt en tussen vierkante haken]} weergegeven.

\item Een uitspraak of proost wordt \quotes{cursief en tussen aanhalingstekens} weergegeven 

\item Een \textbf{kaart} is een speelkaart die gebruikt wordt om te Fritsen.

\item \label{definitie:waarden_kaart} De waarde van de \textbf{kaarten} in oplopende volgorde zijn \kaart{2}, \kaart{3}, \kaart{4}, \kaart{5}, \kaart{6}, \kaart{7}, \kaart{8}, \kaart{9}, \kaart{10}, \kaart{boer}, \kaart{vrouw}, \kaart{koning} en \kaart{aas}. De rest van de \textbf{kaarten} zijn \kaart{jokers}.

\item De vier typen kaartspelsymbolen of \textbf{symbolen} zijn harten, klaveren, ruiten en schoppen. 

\item \label{definitie:kaarten} \'E\'en \textbf{pak kaarten} bestaat uit: 
    \puntLijst{}
        \item de unieke set \textbf{kaarten} met \textbf{symbolen} schoppen, harten klaveren en ruiten met namen \kaart{2}, \kaart{3}, \kaart{4}, \kaart{5}, \kaart{6}, \kaart{7}, \kaart{8}, \kaart{9}, \kaart{10}, \kaart{boer}, \kaart{vrouw}, \kaart{koning} en \kaart{aas}.
        \item twee jokers en, indien aanwezig, een bridge-kaart.
    \eindPuntLijst{}

\item Een \textbf{stapel} is één of meer op elkaar gelegde \textbf{kaarten}.

\item \label{definitie:fritsen} \Fritsen of \textbf{een Fritsje nemen} is het drinken van minimaal 3 cl bier van minimaal 4 procent alcohol uit een shotglaasje of het drinken van een equivalente hoeveelheid alcohol uit een ander glas.

\item \label{definitie:fritsen_equivalent}\textbf{Een Fritsje des nemen}, \textbf{een uittreefritsje nemen}, \textbf{een Lisaatje nemen}, \textbf{een Kimmetje nemen}, \textbf{een Jorisje nemen} en \textbf{een Victortje nemen}  zijn equivalent aan \Fritsen en \textbf{een Fritsje nemen}. 

\item \label{definitie:dubbelfritsen} \textbf{Dubbelfritsen} en \textbf{een dubbele Frits nemen} zijn equivalent aan twee keer \FritsenN.

\item \label{definitie:dubbelfritsen_equivalent} \textbf{Een dubbele Kim nemen}, \textbf{een Erikje nemen}, \textbf{een Thierry'tje nemen} en \\ \textbf{een Caroline'tje nemen} zijn equivalent aan \textbf{dubbelfritsen} en \textbf{een dubbele Frits nemen}.

\item \AlleSpelers zijn diegenen die momenteel aan het Fritsen zijn.

\item \EenSpeler is één van \alleSpelersN. 

\item De \huidigeSpeler is diegene die momenteel aan de beurt\footnotemark[1] is.  

\item Een \andereSpeler of de \andereSpelers zijn diegene(n) die momenteel niet aan de beurt\footnotemark[1] zijn. 

\item De \vorigeSpeler van \eenSpeler is diegene die meteen voor diegene aan de beurt\footnotemark[1] was. 

\item De \volgendeSpeler van \eenSpeler is diegene die meteen na diegene aan de beurt\footnotemark[1] is. 

\item De \medeSpeler van \eenSpeler is \'e\'en van \alleSpelers behalve diegene.

\item De \medeSpelers van \eenSpeler zijn \alleSpelers behalve diegene.

\item \Frits is diegene die het meest op \textit{Frits Kastelein}\footnotemark[2] lijkt volgens meer dan de helft van \alleSpelersN\footnotemark[3].

\item \Willem is diegene die het meest op \textit{Willem Vaandrager}\footnotemark[2] lijkt volgens meer dan de helft van \alleSpelersN\footnotemark[3].

\item \label{definitie:uitfritsen} \textbf{Uitfritsen} is het neerleggen van de laatste kaart(en) van \eenSpelerN.

\item  \label{definitie:uitfritsgarantie} \textbf{Uitfritsgarantie} hebben is het niet meer nieuwe kaarten hoeven te pakken en elke beurt\footnotemark[1] \'e\'en of twee kaarten kunnen wegleggen.

\item \label{definitie:uitklaver} \textbf{Uitklaveren} is het \textbf{uitfritsen} met een \kaart{klaveren 3}\footnotemark[4].

\item \textbf{Dichtfritsen} is een \textbf{stapel} compleet dichtleggen, d.w.z., op die \textbf{stapel} kan alleen nog maar een \kaart{9} worden gelegd.

\item \label{definitie:kanonskogel} Een \textbf{kanonskogel} is een shotglaasje van minimaal 3 cl met 50 procent Sambuca en 50 procent Goldstrike met, mits aanwezig, drie druppels tabasco.

\eindABCLijst

\footnotetext[1]{zie het stroomdiagram op de achterkant van dit document}
\footnotetext[2]{zie de foto's op de achterkant van dit document}
\footnotetext[3]{gebruik bij twijfel regel \ref{regel:beslissingCriteria} op pagina \pageref{regel:beslissingCriteria} }
\footnotetext[4]{zie regels \ref{zet:Jesse_Thierry}, \ref{zet:Jesse_Thierry_2} en \ref{regel:Erik_Thierry} op pagina \pageref{regel:Erik_Thierry}.}