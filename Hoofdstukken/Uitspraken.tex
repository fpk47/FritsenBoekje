\newpage
\drawBar{}
\hoofdstuk{6}{Uitspraken}

\

\beginABCLijst{4}
\item Een \quotes{Olafje doen} is meer dan vijf keer \textbf{vuil Fritsen}\footnotemark[1].

\item Een \quotes{soepele Frits} is het opleggen van een kaart op een kaart van hetzelfde symbool met \'e\'en waarde lager. Bijvoorbeeld, een \kaart{klaveren 7} op een \kaart{klaveren 6}.

\item Een \quotes{stroeve Frits} is een kaart opleggen die het de \andereSpelers moeilijk maakt.

\item Een \quotes{kartelfrits} is het opleggen van een kaart ongelijk aan een \kaart{6} op een \kaart{vrouw}.

\item Een \quotes{lavendelfrits} is het opleggen van een \kaart{6} op een kaart ongelijk aan een \kaart{vrouw}.

\item Een \quotes{tactical Frits} is een kaart opleggen die een waarde heeft van tenminste 9 hoger. Bijvoorbeeld een \kaart{harten vrouw}, \kaart{harten koning} leggen.

\item Een \quotes{eerste Frits} is het opleggen van de eerste \kaart{koning}.

\item Een \quotes{eerste Kim} is het opleggen van de eerste \kaart{vrouw}.

\item \quotes{Bijfritsen} is het inschenken van shotglaasjes met de drank die op dat moment gebruikt wordt om te Fritsen.

\item Een \quotes{Otten plegen} is het inwisselen van je kaarten tijdens het \textbf{vuil Fritsen}\footnotemark[1] wanneer één van je kaarten die je inwisselt een \kaart{6} is. 

\item Een \quotes{soepele start} is het uitvoeren van een \textbf{offerfrits}\footnotemark[2] door \Willem als hij/zij nog geen kaart(en) heeft opgelegd.

\item Een \quotes{eiken Frits} is het \textbf{uitfritsen}\footnotemark[3] in twee beurten\footnotemark[4].

\item Een \quotes{platina Frits} is het \textbf{uitfritsen}\footnotemark[3] in drie beurten\footnotemark[4].

\item Een \quotes{diamanten Frits} is het \textbf{uitfritsen}\footnotemark[3] in vier beurten\footnotemark[4].

\item Een \quotes{gouden Frits} is het \textbf{uitfritsen}\footnotemark[3] in vijf beurten\footnotemark[4].

\item Een \quotes{zilveren Frits} is het \textbf{uitfritsen}\footnotemark[3] met \'e\'en keer kaarten pakken.

\item Een \quotes{bronzen Frits} is het \textbf{uitfritsen}\footnotemark[3] met twee keer kaarten pakken.
\eindABCLijst

\footnotetext[1]{\textbf{vuil Fritsen} is het drinken en daarna inwisselen van je vijf startkaarten (zie pagina \pageref{regel:zet_vuil_fritsen})} 
\footnotetext[2]{zie zet \ref{zet:offer_frits} op pagina \pageref{zet:offer_frits}} 
\footnotetext[3]{\textbf{uitfritsen} is het neerleggen van je laatste kaart(en) (zie definitie \ref{item:uitfritsen})}
\footnotetext[4]{zie het stroomdiagram op de achterkant van dit document}