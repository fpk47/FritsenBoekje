\hoofdstuk{4}{Uitspraken}

\beginABCLijstKlein{4}
\item Een \quotes{Olafje doen} is meer dan vijf keer \textbf{vuil Fritsen}\footnotemark[2].

\item Een \quotes{soepele Frits} is het opleggen van een kaart op een kaart van hetzelfde symbool met \'e\'en waarde lager. Bijvoorbeeld, een \kaart{klaveren 7} op een \kaart{klaveren 6}.

\item Een \quotes{stroeve Frits} is het opleggen van een kaart met een waarde van tenminste 9 hoger. Bijvoorbeeld een \kaart{harten koning} op een \kaart{harten 3} leggen.

\item Een \quotes{gladiolen Frits} is het opleggen van een kaart die het \andereSpelers moeilijk maakt om \textbf{uit te fritsen}\footnotemark[3] ten koste van de kans om zelf \textbf{uit te fritsen}\footnotemark[3].

\item Een \quotes{uitfritsloze situatie} is een situatie waarbij \eenSpeler in het huidige potje hoogstwaarschijnlijk als enige nog kaarten overhoudt.

\item Een \quotes{tactical Frits} is het opleggen van een kaart die het \andereSpelers moeilijk maakt.

\item Een \quotes{lavendelfrits} is het opleggen van een \kaart{6} op een kaart ongelijk aan een \kaart{vrouw}.

\item Een \quotes{eerste Kim} is het opleggen van de eerste \kaart{vrouw}.

\item Een \quotes{vierde Kim} is het opleggen van een \kaart{vrouw} op een stapel met drie achtereenvolgende \kaart{vrouwen}.

\item \quotes{Bijfritsen} is het inschenken van shotglaasjes die gebruikt worden om te Fritsen met drank.

\item Een \quotes{soepele start} is het opleggen van een kaart op een kaart van hetzelfde symbool met \'e\'en waarde lager door \Willem als die speler nog geen kaart(en) heeft opgelegd.

\item Een \quotes{stroeve start} is het opleggen van een kaart met een waarde van tenminste 9 hoger door \Willem als die speler nog geen kaart(en) heeft opgelegd.

\item Een \quotes{tactical start} is het opleggen van een kaart die het \andereSpelers moeilijk maakt door \Willem als die speler nog geen kaart(en) heeft opgelegd.

\item Een \quotes{offer start} is het uitvoeren van een \textbf{offerfrits}\footnotemark[3] door \Willem als die speler nog geen kaart(en) heeft opgelegd.

\item Een \quotes{eiken Frits} is het \textbf{uitfritsen}\footnotemark[3] in twee beurten\footnotemark[4].

\item Een \quotes{platina Frits} is het \textbf{uitfritsen}\footnotemark[3] in drie beurten\footnotemark[4].

\item Een \quotes{diamanten Frits} is het \textbf{uitfritsen}\footnotemark[3] in vier beurten\footnotemark[4].

\item Een \quotes{gouden Frits} is het \textbf{uitfritsen}\footnotemark[3] in vijf beurten\footnotemark[4].

\item Een \quotes{zilveren Frits} is het \textbf{uitfritsen}\footnotemark[3] in zes of zeven beurten.

\item Een \quotes{bronzen Frits} is het \textbf{uitfritsen}\footnotemark[3] in acht of negen beurten.
\eindABCLijst

% LET OP
% Hoofdstuk Varianten heeft al footnotetext[1], daarom beginnen ze hier met [2]

\footnotetext[2]{\textbf{vuil Fritsen} is het drinken en daarna inwisselen van je vijf startkaarten (zie pagina \pageref{regel:zet_vuil_fritsen})}
\footnotetext[3]{\textbf{uitfritsen} is het neerleggen van je laatste \textbf{kaart(en)} (zie definitie \ref{item:uitfritsen} op pagina \pageref{item:uitfritsen})}
\footnotetext[4]{zie het stroomdiagram op de achterkant van dit document}